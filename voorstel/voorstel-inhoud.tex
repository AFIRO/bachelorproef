%---------- Inleiding ---------------------------------------------------------

\section{Inleiding}%
\label{sec:inleiding}

Deze bachelorproef richt zich op het verbeteren van process governance door middel van process monitoring en orchestration, specifiek binnen het bedrijf Liantis, een grote Belgische onderneming actief in de HR sector. Liantis biedt oplossingen in kader van payroll, preventie/welzijn op het werk en ondersteuning aan zelfstandigen. Omdat het bedrijf een wildgroei aan processen heeft die steeds complexer en dynamischer worden, heeft het Liantis moeite om een consistent en betrouwbaar beheer van zijn processen te waarborgen. Vooral door beperkte mogelijkheden om real-time inzicht te krijgen in de prestaties van hun medewerkers, ontstaan er regelmatig inefficiënties en compliance-uitdagingen. Voor business-managers en -analisten binnen dit bedrijf vormt dit een knelpunt bij het waarborgen van de procesbetrouwbaarheid en het naleven van interne en externe regelgeving. De centrale onderzoeksvraag luidt dan ook "Hoe kan process monitoring en orchestration bijdragen aan een effectieve ondersteuning van process governance?" en dit praktisch toegepast binnen het specifieke geval van Liantis. Een bijkomende vraag is "Wat zijn de technische uitdagingen bij de implementatie van process monitoring en orchestration systemen bij Belgische bedrijf met zowel inhouse als externe software?" omdat deze natuurlijk zal voortvloeien vanuit de casus Liantis.

Het doel van deze bachelorproef is om een concreet raamwerk en aanbevelingen te ontwikkelen voor het implementeren van process monitoring en orchestration om zo de process governance binnen Liantis te optimaliseren. Verder zal het een proof-of-concept realizeren waarbij we dit toepassen op een specifiek gekozen process. Dit zal dan gebruikt worden om de relevante stakeholders te overtuigen om dit verder uit te rollen naar andere processen.  Hiervoor zal eerst een uitgebreide literatuurstudie worden uitgevoerd om inzicht te krijgen in bestaande methoden en tools op het gebied van process monitoring en orchestration. Vervolgens wordt er een casestudie uitgevoerd binnen het bedrijf om de huidige knelpunten in kaart te brengen en de impact van mogelijke oplossingen te evalueren. Op basis hiervan zal een proof-of-concept ontwikkeld worden waarbij we aantonen hoe dit praktisch een specifiek gekozen complex bedrijfsprocess zal ondersteunen. Het beoogde eindresultaat voor een succesvolle bachelorproef bestaat uit een gedetailleerd implementatieplan dat de IT-afdeling van Liantis helpt om monitoring en orchestration op een effectieve manier in de governance-structuur te integreren met een proof-of-concept die kan dienen als sjabloon voor verdere implementatie. 

%---------- Stand van zaken ---------------------------------------------------

\section{Stand Van Zaken}%
\label{sec:stand_van_zaken}
\subsection{Business Process Management en Governance}

Het domein van business process management (BPM) en governance richt zich op het verbeteren van bedrijfsprocessen door middel van monitoring, analyse, en optimalisatie. BPM biedt organisaties de mogelijkheid om processen te stroomlijnen en deze af te stemmen op de strategische doelen van de organisatie \autocite{Dumas2018}. Binnen deze context spelen process monitoring en orchestration een cruciale rol. Deze technologieën maken het mogelijk om bedrijfsprocessen op een efficiënte en geautomatiseerde manier te beheren en aan te passen \autocite{Weske2019}.

\subsection{Monitoring, Mining en Orchestration}

Process monitoring is essentieel binnen BPM omdat het organisaties near-realtime inzicht geeft in de prestaties van hun bedrijfsprocessen, waardoor afwijkingen of inefficiënties snel kunnen worden geïdentificeerd \autocite{Janiesch2012}. Dit is dan voedingsbodem voor process mining, waarbij we de logs uit de monitoring gebruiken om tot analytische inzichten te komen. Process mining biedt niet alleen inzicht in het huidige verloop van processen maar maakt ook een data-gedreven optimalisatie mogelijk door onregelmatigheden bloot te leggen die vaak onzichtbaar blijven bij traditionele monitoring \autocite{Aalst2016}. Daarnaast kan Business Activity Monitoring (BAM) organisaties helpen om niet alleen retrospectief maar ook proactief op afwijkingen te reageren, wat waardevol is voor processen met hoge compliance-eisen \autocite{Janiesch2012}. Process orchestration gaat een stap verder dan monitoring door bedrijfsprocessen te automatiseren en te coördineren op basis van vastgelegde beleidslijnen en workflows. Dit is vooral nuttig voor organisaties die streven naar consistentie en compliance in hun bedrijfsvoering \autocite{Weske2019}. 

\subsection{Open Vragen}

De grote open vraag die heeft geleid tot dit onderzoek is de technische uitdaging van het implementeren van deze systemen binnen een groot bedrijf zoals Liantis. Net zoals veel bedrijven is Liantis immers organisch gegroeid en bevat het niet enkel veel inhouse legacy systemen, maar ook aangekochte systemen. Dergelijke monitoring en orchestratie systemen moeten op een agnostische manier kunnen omgaan met de output van allerhande processen en systemen. De implementatie van deze systemen in een bedrijfscontext brengt dus verschillende technische uitdagingen met zich mee die vaak custom oplossingen vereisen. Vakliteratuur geeft een aantal interessante opties zoals het gebruik van middleware \autocite{Weber2018}, maar er is een zeer duidelijke lacune te zien binnen het domein om te onderzoeken. 

\subsection{Vergelijkbare Studies en Unieke Waarde}

De waarde van dit domein is al zeker en vast bewezen door Vergelijkbare studies binnen grote organisaties \autocite{Harmon2014} die benadrukken dat de implementatie van geavanceerde monitoring- en orchestrationtechnieken helpen om consistent aan governance- en compliance-vereisten te voldoen. Dit onderzoek biedt echter een unieke bijdrage door zich te richten op de praktische toepasbaarheid van deze technieken bij een groot Belgische bedrijf wiens IT-context breed toepasbaar is. Door dit uit te werken zal er een praktisch raamwerk bestaan voor de implementatie van dergelijke systemen binnen een bedrijf met dergelijk profiel met proof-of-concept dat kan dienen als bluedruk voor veel Belgische bedrijven. 

%---------- Methodologie ------------------------------------------------------
\section{Methodologie}%
\label{sec:methodologie}

Om de onderzoeksvraag te beantwoorden en een effectieve aanpak te ontwikkelen voor de implementatie van process monitoring en orchestration binnen Liantis, wordt gebruikgemaakt van een gefaseerde onderzoeksmethodologie die literatuurstudie, case study-analyse, requirements-analyse en een proof-of-concept ontwikkeling omvat. Deze combinatie van technieken waarborgt zowel theoretische onderbouwing als technische diepgang, wat essentieel is voor het creëren van een valide en werkbaar implementatiemodel.

\subsection{Fase 1: Literatuurstudie (2 weken)}

De eerste fase bestaat uit een gedetailleerde literatuurstudie. Hierbij worden academische artikelen en technische bronnen over business process management, monitoring, orchestration, en implementatie-uitdagingen grondig geanalyseerd. Deze studie heeft als doel om een diepgaand inzicht te verkrijgen in de huidige stand van zaken en de typische uitdagingen en succesfactoren binnen BPM-implementaties voor process governance. De bevindingen uit de literatuurstudie vormen de theoretische basis van het onderzoek en dienen als referentiekader voor de verdere fasen. \\

\textbf{Deliverable:} een theoretisch component met gedocumenteerde inzichten over BPM-standaarden, technische obstakels, en oplossingen voor de organisatie.

\subsection{Fase 2: Requirements-analyse binnen Liantis (2 weken)}

Om de specifieke behoeften en uitdagingen van de casusorganisatie vast te stellen, wordt een requirements-analyse uitgevoerd door middel van gestructureerde workshops met stakeholders binnen de organisatie, zoals IT-managers, procesanalisten, en compliance-officers. Verder wordt de bestaande IT-infrastructuur bekeken om te kijken hoe we dit het beste inpassen om de impact minimaal te houden. We bepalen hier ook het complexe process dat in scope zal zijn voor onze proof-of-concept. \\

\textbf{Deliverable:} een requirementsrapport met gedetailleerde, gevalideerde vereisten voor de implementatie.

\subsection{Fase 3: Technische en Functionele Analyse (2 weken)}

Na de requirements-analyse wordt een technische analyse uitgevoerd om geschikte tools en technologieën te selecteren. Hierbij zoeken we naar een process documentatie tool waarmee we kunnen integreren als bron van waarheid, modeleren we een architectuur voor deze systemen, schrijven we een contract uit dat alle systemen gaan gebruiken voor de monitoring en orchestratie en voeren we de analyse uit voor de proof-of-concept.  \\

\textbf{Deliverable:} design documenten voor de process monitoring en orchestration tools met implementatieplan.

\subsection{Fase 4: Proof-of-Concept Ontwikkeling en Validatie (1 maand)}

In deze fase wordt een proof-of-concept ontwikkeld om de haalbaarheid van het voorgestelde implementatiemodel te testen. De PoC omvat de integratie van de monitoring- en orchestration systeem met een voorbeeldproces van de Liantis. Dit proces wordt gesimuleerd met representatieve data om de configuratie van monitoring en event-triggered orchestration te testen. Deze proof-of-concept zal dan gebruikt worden stakeholders te overtuigen en als blauwdruk voor verdere processen. \\

\textbf{Deliverable:} een gedetailleerd PoC-rapport en werkend prototype dat het systeem demonstreert.

\subsection{Tijdschema}

\begin{table}[h!]
\centering
\begin{tabular}{|p{6cm}|p{2.5cm}|p{5cm}|}
\hline
\textbf{Fase} & \textbf{Duur} & \textbf{Deliverables} \\
\hline
Literatuurstudie & 2 weken & Theoretisch component \\
\hline
Requirements-analyse & 2 weken & Requirementsrapport \\
\hline
Technische en Functionele Analyse & 2 weken & Design documenten en implementatieplan. \\
\hline
Proof-of-Concept Ontwikkeling en Validatie & 1 maand & PoC-rapport en werkend prototype \\
\hline
\end{tabular}
\caption{Tijdschema van de verschillende onderzoeksfasen en deliverables.}
\end{table}

%---------- Verwachte resultaten ----------------------------------------------
\section{Verwacht resultaat, conclusie}%
\label{sec:verwachte_resultaten}
De verwachting binnen dit onderzoek is dat er een implementatieplan zal ontstaan die toelaat om process governance te waarborgen binnen Liantis door te steunen op een sterk framework van monitoring en orchestration met een proof-of-concept om stakeholders te overtuigen. Dit zal dan fungeren als verdere impetus voor het uitwerken van dit systeem binnen de organisatie met als gevolg consistentere processen die makkelijker geoptimaliseerd kunnen worden.

Verder is het mijn verwachting dat de resultaten uit dit onderzoek toepasbaar zullen zijn op bedrijven die een gelijkaardig profiel hebben als Liantis. Liantis is binnen zijn IT-context zeker geen uitzondering en het is mijn doel om binnen dit onderzoek Liantis als case study te gebruiken om zo een framework te ontwikkelen waarmee veel meer bedrijven gediend zijn.

