%---------- Inleiding ---------------------------------------------------------

\section{Inleiding}%
\label{sec:inleiding}

\subsection{Context}

Dit werk richt zich op het verbeteren van process governance door middel van process monitoring en orchestration, specifiek binnen het bedrijf Liantis, een grote Belgische onderneming actief in de HR-sector. Liantis biedt oplossingen in kader van payroll, preventie/welzijn op het werk en ondersteuning aan zelfstandigen. Omdat het bedrijf een wildgroei aan processen heeft die steeds complexer en dynamischer worden, heeft Liantis moeite om consistent en betrouwbaar beheer van zijn processen te waarborgen. Door zijn veelvoud aan producten is Liantis ook vaak aangewezen op custom development. Vooral door beperkte mogelijkheden om real-time inzicht te krijgen in de prestaties van zijn medewerkers binnen deze custom solutions, ontstaan er regelmatig inefficiënties en compliance-uitdagingen. Voor business-managers en analisten binnen het bedrijf vormt dit een knelpunt bij het waarborgen van de procesbetrouwbaarheid en het naleven van interne en externe regelgeving. 

\subsection{Onderzoeksvraag en Deelvragen}
De centrale onderzoeksvraag luidt dan ook \textbf{"Wat zijn de requirements waaraan een systeem moet voldoen om custom process monitoring en orchestration te implementeren binnen een bedrijf met veel custom development solutions?"} en dit praktisch toegepast binnen de specifieke case van Liantis. Door dit zodanig specifiek toe te passen kunnen we de deelvragen heel concreet maken voor zowel het probleemdomein als het oplossingsdomein.

\textbf{Deelvragen:}
\begin{itemize}
    \item \textbf{Probleemdomein:}
    \begin{enumerate}
        \item Wat zijn de theoretisch high-level requirements waaraan dit systeem moet voldoen?
        \item Wat zijn de mogelijkheden die bestaande process monitoring en orchestration software op de markt bieden?
        \item Wat zijn de specifieke noden van Liantis qua process monitoring en orchestration?
        \item Welke technische en architecturale uitdagingen zijn er bij de implementatie van dergelijke systemen in een bedrijfsomgeving met veel custom development?
    \end{enumerate}
    
    \item \textbf{Oplossingsdomein:}
    \begin{enumerate}
        \item Hoe implementeren we dergelijke systemen in een bedrijfsomgeving met veel custom development zoals Liantis?
        \item Aan welke criteria moet dergelijk systeem voldoen om succesvolle process monitoring en orchestration te bereiken binnen het bedrijf?
    \end{enumerate}
\end{itemize}

\subsection{Doel en Resultaat}

Het doel van deze bachelorproef is om een analyse te maken van process monitoring en orchestration binnen de huidige IT-markt en dit naast de specifieke noden van Liantis te zetten om zo een solution te maken voldoet voor hun doeleinden. Verder zal het een proof-of-concept realiseren in de vorm van een integreerbare library die in zowel legacy als nieuwe solutions kan werken en die procesmonitoring logs stuurt naar een tool die dit allemaal bijhoudt in leesbare vorm waarop rapportage kan gedaan worden en die in kader van orchestration taken kan genereren voor medewerkers. We gaan deze proof-of-concept dan toepassen op een specifiek gekozen proces om data te genereren. Dit zal dan gebruikt worden om de relevante stakeholders te overtuigen om dit verder uit te bouwen en uit te rollen naar andere processen.  Hiervoor zal eerst een uitgebreide literatuurstudie worden uitgevoerd om inzicht te krijgen in bestaande methodieken en theoretische achterslag op het gebied van process monitoring en orchestration. Dit wordt dan gevolgd door een marktonderzoek waarbij we de mogelijkheden van bestaande tools op de markt bekijken. Vervolgens wordt er een casestudie uitgevoerd binnen het bedrijf om de huidige knelpunten en noden in kaart te brengen. Op basis hiervan zal een proof-of-concept ontworpen en ontwikkeld worden waarbij we aantonen hoe dit een praktisch en specifiek gekozen bedrijfsproces zal ondersteunen. Het beoogde eindresultaat voor een succesvolle bachelorproef bestaat uit een gedetailleerd implementatieplan met proof-of-concept dat de IT-afdeling van Liantis helpt om monitoring en orchestration op een effectieve manier in de governance-structuur te integreren.

%---------- Stand van zaken ---------------------------------------------------

\section{Stand Van Zaken}%
\label{sec:stand_van_zaken}
\subsection{Business Process Management en Governance}

Het domein van business process management en governance richt zich op het verbeteren van bedrijfsprocessen door middel van monitoring, analyse, en optimalisatie. BPM biedt organisaties de mogelijkheid om processen te stroomlijnen en deze af te stemmen op de strategische doelen van de organisatie \autocite{Dumas2018}. Binnen deze context spelen process monitoring en orchestration een cruciale rol. Deze technologieën maken het mogelijk om bedrijfsprocessen op een efficiënte en geautomatiseerde manier te beheren en aan te passen \autocite{Weske2019}.

\subsection{Monitoring en Orchestration}

Process monitoring is essentieel binnen BPM omdat het organisaties near-realtime inzicht geeft in de prestaties van hun bedrijfsprocessen, waardoor afwijkingen of inefficiënties snel kunnen worden geïdentificeerd \autocite{Janiesch2012}. Dit is dan voedingsbodem voor process mining, waarbij we de logs uit de monitoring gebruiken om tot analytische inzichten te komen. Process mining biedt niet alleen inzicht in het huidige verloop van processen, maar maakt ook een data-gedreven optimalisatie mogelijk door onregelmatigheden bloot te leggen die vaak onzichtbaar blijven bij traditionele monitoring \autocite{Aalst2016}. Daarnaast kan Business Activity Monitoring organisaties helpen om niet alleen retrospectief, maar ook proactief op afwijkingen te reageren, wat waardevol is voor processen met hoge compliance-eisen \autocite{Janiesch2012}. Process orchestration gaat een stap verder dan monitoring door bedrijfsprocessen te automatiseren en te coördineren op basis van vastgelegde beleidslijnen en workflows. Dit is vooral nuttig voor organisaties die streven naar consistentie en compliance in hun bedrijfsvoering \autocite{Weske2019}. 

\subsection{Open Vragen}

De open vraag die heeft geleid tot dit onderzoek is de technische uitdaging van het implementeren van deze systemen binnen een bedrijf zoals Liantis met veel custom development. Net zoals veel bedrijven is Liantis immers organisch gegroeid en bevat het veel legacy inhouse solutions die op elkaar inspelen. Dergelijke monitoring en orkestratie systemen moeten op een agnostische manier kunnen omgaan met de output van allerhande processen en systemen. De implementatie van deze systemen in een bedrijfscontext brengt dus verschillende technische uitdagingen met zich mee die vaak opnieuw custom oplossingen vereisen. Vakliteratuur geeft een aantal interessante opties zoals het gebruik van middleware \autocite{Weber2018}, maar er is een zeer duidelijke lacune binnen het domein dat interessant zou zijn om te onderzoeken. 

\subsection{Vergelijkbare Studies en Unieke Waarde}

De waarde van dit domein is al vast en zeker bewezen door vergelijkbare studies binnen grote organisaties \autocite{Harmon2014} die benadrukken dat de implementatie van geavanceerde monitoring- en orchestration technieken helpen om consistent aan governance- en compliance-vereisten te voldoen. Dit onderzoek biedt echter een unieke bijdrage door zich te richten op de praktische toepasbaarheid van deze technieken bij een groot Belgische bedrijf wiens IT-context breed toepasbaar is. Door dit uit te werken zal er een praktisch raamwerk bestaan voor de implementatie van dergelijke systemen binnen een bedrijf met dergelijk profiel met proof-of-concept dat kan dienen als blauwdruk voor veel Belgische bedrijven. 

%---------- Methodologie ------------------------------------------------------
\section{Methodologie}%
\label{sec:methodologie}

Om de onderzoeksvraag te beantwoorden en een effectieve aanpak te ontwikkelen voor de implementatie van process monitoring en orchestration binnen Liantis, wordt gebruik gemaakt van een gefaseerde onderzoeksmethodologie die literatuurstudie, marktonderzoek, case studieanalyse, requirements-analyse en een proof-of-concept ontwikkeling omvat. Deze combinatie van technieken waarborgt zowel theoretische onderbouwing als technische diepgang, wat essentieel is voor het creëren van een valide en werkbaar implementatiemodel.

\subsection{Fase 1: Literatuurstudie (4 weken) (Probleemdomein)}

De eerste fase bestaat uit een gedetailleerde literatuurstudie en marktonderzoek. Hierbij worden academische artikelen en technische bronnen over business process management, monitoring, orchestration, en implementatie-uitdagingen grondig geanalyseerd. Deze studie heeft als doel om een diepgaand inzicht te verkrijgen in de huidige stand van zaken van het domein, de mogelijkheden binnen de huidige markt, de typische uitdagingen en succesfactoren binnen BPM-implementaties voor process governance. De bevindingen uit de literatuurstudie vormen de theoretische basis van het onderzoek en dienen als referentiekader voor de verdere fasen. \\

\textbf{Deliverable:} een theoretisch component met gedocumenteerde inzichten over BPM-standaarden, technische obstakels en oplossingen op de markt voor de organisatie.

\subsection{Fase 2: Requirements-analyse binnen Liantis (2 weken) (Probleemdomein)}

Om de specifieke behoeften en uitdagingen van de casusorganisatie vast te stellen, wordt een requirements-analyse uitgevoerd door middel van gestructureerde workshops met stakeholders binnen de organisatie, zoals IT-managers, procesanalisten en compliance-officers. Verder wordt de bestaande IT-infrastructuur bekeken om te kijken hoe we dit het beste inpassen om de impact minimaal te houden. We bepalen hier ook het complexe proces dat in scope zal zijn voor onze proof-of-concept. \\

\textbf{Deliverable:} een requirementsrapport met gedetailleerde, gevalideerde vereisten voor de implementatie.

\subsection{Fase 3: Technische en Functionele Analyse (2 weken) (Oplossingsdomein)}

Na de requirements-analyse wordt een technische analyse uitgevoerd om geschikte tools en technologieën te selecteren. Hierbij zoeken we naar een proces documentatie tool waarmee we kunnen integreren als bron van waarheid, modelleren we een architectuur voor deze systemen, schrijven we een contract uit dat alle systemen gaan gebruiken voor de monitoring en orkestratie en voeren we de analyse uit voor de proof-of-concept.  \\

\textbf{Deliverable:} design documenten voor het process monitoring en orchestration systeem met implementatieplan.

\subsection{Fase 4: Proof-of-Concept Ontwikkeling en Validatie (4 weken) (Oplossingsdomein)}

In deze fase wordt een proof-of-concept ontwikkeld om de haalbaarheid van het voorgestelde implementatiemodel te testen. De proof-of-concept omvat de integratie van het monitoring- en orchestration systeem met een voorbeeldproces van de Liantis. Dit proces wordt gesimuleerd met representatieve data om de configuratie van monitoring en event-triggered orchestration te testen. Uit de data zullen we dan eerste conclusies en rapportage kunnen opbouwen voor verdere iteratieve verbeteringen en aanpassingen. Deze proof-of-concept zal dan gebruikt worden om stakeholders te overtuigen en als blauwdruk dienen voor de implementatie van dit systeem bij verdere processen. \\

\textbf{Deliverable:} een gedetailleerd proof-of-concept rapport en werkend prototype dat het systeem demonstreert.

%---------- Verwachte resultaten ----------------------------------------------
\section{Verwachtingen}%
\label{sec:verwachtingen}
De verwachting binnen dit onderzoek is dat er een implementatieplan zal ontstaan die toelaat om process governance te waarborgen binnen Liantis door te steunen op een sterk framework van monitoring en orchestration met een proof-of-concept om stakeholders te overtuigen. Dit zal dan fungeren als verdere voedingsbodem voor het uitwerken van dit systeem binnen de organisatie met als gevolg consistentere processen die makkelijker geoptimaliseerd kunnen worden.

Verder is het mijn verwachting dat de resultaten uit dit onderzoek toepasbaar zullen zijn op bedrijven die een gelijkaardig profiel hebben als Liantis. Liantis is binnen zijn IT-context zeker geen uitzondering en het is mijn doel om binnen dit onderzoek Liantis als casestudy te gebruiken om zo een framework te ontwikkelen waarmee veel meer bedrijven gediend zijn.

