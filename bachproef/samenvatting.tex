\IfLanguageName{english}{%
\selectlanguage{dutch}
\chapter*{Samenvatting}
Dit werk onderzoekt hoe we proces monitoring en orkestratie kunnen implementeren binnen een Belgisch bedrijf met een IT-context waarbij er veel legacy toepassingen zijn. De centrale onderzoeksvraag is: ”Wat zijn de vereisten waaraan een systeem moet voldoen en welke software architectuur is gepast om custom proces monitoring en orkestratie te implementeren binnen een bedrijf met veel custom development applicaties?”.
Dit onderzoek beoogt een praktisch raamwerk te ontwikkelen met een proof-of-concept dat de toepassing van deze technologieën in een governance-structuur beschrijft en faciliteert. Verder wordt dit raamwerk praktisch toegepast op het bedrijf Liantis met een eerste proof-of-concept waarin een specifiek proces wordt ondersteund met monitoring en orkestratie. Theoretisch en marktonderzoek wordt gebruikt om te ontdekken wat er gangbaar is binnen het domein, waarnaar we dit toetsen aan de noden van het casusbedrijf voor een praktische requirements analyse en proof-of-concept. Voor de proof-of-concept wordt vervolgens een simulatie van het test-proces uitgevoerd om zo een waarheidsgetrouw resultaat te leveren waaruit wij verdere conclusies kunnen trekken rond de praktische toepasbaarheid van het systeem en verdere iteratieve verbeteringen.
\selectlanguage{english}
}{}


\chapter*{\IfLanguageName{dutch}{Samenvatting}{Abstract}}

\lipsum[1-4]
