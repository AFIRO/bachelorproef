\chapter*{\IfLanguageName{dutch}{Samenvatting}{Abstract}}
\selectlanguage{dutch}
\label{ch:abstract}

Dit werk onderzoekt hoe proces monitoring en orkestratie geïmplementeerd kan worden binnen een Belgisch bedrijf met een IT-context waarbij er veel legacy toepassingen zijn. Het onderzoek kadert binnen de toenemende nood aan effectieve governance in complexe business-omgevingen, waar handmatige monitoring vaak ontoereikend is om problemen snel te detecteren en op te lossen. Verder kadert het zich in de groeiende vraag van bedrijven om hun medewerkers zo effectief mogelijk in te zetten via automatisatie. De centrale onderzoeksvraag is: \textbf{”Wat zijn de vereisten waaraan een systeem moet voldoen en welke software architectuur is gepast om custom proces monitoring en orkestratie te implementeren binnen een bedrijf met veel custom development applicaties?”}.\newline

Methodologisch wordt een combinatie van literatuurstudie rond het domein, marktonderzoek en casestudie-analyse binnen het casusbedrijf Liantis gebruikt om het probleemdomein op te bouwen en een oplossing te generen. Er wordt gekeken naar de theorie en de markt om te onderzoeken wat er gangbaar is binnen het domein, waarnaar dit getoetst wordt aan de noden van Liantis voor een praktische functionele requirementsanalyse, technische analyse en niet-functionele analyse. \newline

Op basis van deze analyse wordt een theoretisch en proof-of-concept systeem ontworpen, waarna de proof-of-concept wordt gebouwd. Deze proof-of-concept wordt dan toegepast in een simulatie van een bestaand proces van het casusbedrijf om stakeholders te overtuigen van deze aanpak. \newline

Na afloop van onderzoek wordt een werkende proof-of-concept van product geleverd dat uitgebouwd kan worden tot een volledig product dat voldoet aan de eisen voor proces monitoring en orkestratie bij een Belgisch bedrijf met veel custom en legacy toepassingen. Het toont verder aan dat er duidelijk plaats is voor dergelijk product binnen het werkveld en dat de out-of-the-box toepassingen die momenteel voorhanden zijn nog niet voldoen aan alle noden in het Belgisch werkveld. 