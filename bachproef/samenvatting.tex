%%=============================================================================
%% Samenvatting
%%=============================================================================

% TODO: De "abstract" of samenvatting is een kernachtige (~ 1 blz. voor een
% thesis) synthese van het document.
%
% Een goede abstract biedt een kernachtig antwoord op volgende vragen:
%
% 1. Waarover gaat de bachelorproef?
% 2. Waarom heb je er over geschreven?
% 3. Hoe heb je het onderzoek uitgevoerd?
% 4. Wat waren de resultaten? Wat blijkt uit je onderzoek?
% 5. Wat betekenen je resultaten? Wat is de relevantie voor het werkveld?
%
% Daarom bestaat een abstract uit volgende componenten:
%
% - inleiding + kaderen thema
% - probleemstelling
% - (centrale) onderzoeksvraag
% - onderzoeksdoelstelling
% - methodologie
% - resultaten (beperk tot de belangrijkste, relevant voor de onderzoeksvraag)
% - conclusies, aanbevelingen, beperkingen
%
% LET OP! Een samenvatting is GEEN voorwoord!

%%---------- Nederlandse samenvatting -----------------------------------------
%
% TODO: Als je je bachelorproef in het Engels schrijft, moet je eerst een
% Nederlandse samenvatting invoegen. Haal daarvoor onderstaande code uit
% commentaar.
% Wie zijn bachelorproef in het Nederlands schrijft, kan dit negeren, de inhoud
% wordt niet in het document ingevoegd.

\IfLanguageName{english}{%
\selectlanguage{dutch}
\chapter*{Samenvatting}
\lipsum[1-4]
\selectlanguage{english}
}{}

%%---------- Samenvatting -----------------------------------------------------
% De samenvatting in de hoofdtaal van het document
Dit werk onderzoekt hoe we proces monitoring en orkestratie kunnen implementeren binnen een Belgisch bedrijf met een IT-context waarbij er veel legacy toepassingen zijn. De centrale onderzoeksvraag is: ”Wat zijn de vereisten waaraan een systeem moet voldoen en welke software architectuur is gepast om custom proces monitoring en orkestratie te implementeren binnen een bedrijf met veel custom development applicaties?”.
Dit onderzoek beoogt een praktisch raamwerk te ontwikkelen met een proof-of-concept dat de toepassing van deze technologieën in een governance-structuur beschrijft en faciliteert. Verder wordt dit raamwerk praktisch toegepast op het bedrijf Liantis met een eerste proof-of-concept waarin een specifiek proces wordt ondersteund met monitoring en orkestratie. Theoretisch en marktonderzoek wordt gebruikt om te ontdekken wat er gangbaar is binnen het domein, waarnaar we dit toetsen aan de noden van het casusbedrijf voor een praktische requirements analyse en proof-of-concept. Voor de proof-of-concept wordt vervolgens een simulatie van het test-proces uitgevoerd om zo een waarheidsgetrouw resultaat te leveren waaruit wij verdere conclusies kunnen trekken rond de praktische toepasbaarheid van het systeem en verdere iteratieve verbeteringen.



\chapter*{\IfLanguageName{dutch}{Samenvatting}{Abstract}}

\lipsum[1-4]
