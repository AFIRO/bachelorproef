%%=============================================================================
%% Onderzoeksmethode
%%=============================================================================

\chapter{\IfLanguageName{dutch}{Onderzoeksmethode}{Methodology}}%
\label{ch:onderzoeksmethode}

Het onderzoek voor dit werk bestaat uit een vierledige aanpak waarin enerzijds de vragen binnen zowel het probleemdomein als het oplossingsdomein worden opgelost en anderzijds een proof-of-concept wordt onderzocht, ontworpen en gebouwd die voldoet aan functionele en non-functionele vereisten van het casusbedrijf Liantis. \newline

Allereerst wordt er een literatuurstudie gedaan waarin de nodige theoretische achtergrondkennis wordt verzameld om dit project tot een goed einde te brengen. Hierbij wordt eerst onderzocht hoe “process governance” werkt en hoe proces monitoring en orkestratie hierop inhaken om waarde te creëren voor het bedrijf door middel van toonaangevende werken binnen het domein. De meer technisch kant van het verhaal wordt verder onderzocht door de documentatie van reeds bestaande producten en proof-of-concepts te bekijken om te zien hoe deze opgebouwd zijn.\newline

De tweede fase bestaat uit een requirements en technische analyse om te ontdekken wat het casusbedrijf nodig heeft en wat zowel de functionele als niet-functionele vereisten zijn. Dit gebeurt enerzijds door middel van een workshop waarin de theoretische use-cases van een aantal stakeholders vanuit de business kant worden gecapteerd en anderzijds op basis van een sparring sessie met een systeem architect waarin wordt bekeken hoe dergelijk theoretisch systeem past binnen de architectuur van het bedrijf.\newline

De derde fase bestaat uit het ontwerpen en implementeren van de eigenlijke proof-of-concept. Op basis van de voorgenoemde vereisten ontwerpen we een systeem dat voldoet en documenteren we dit in de nodige deliverables zoals een architecturaal-diagram, een domeinmodel, een entiteit relatie diagram en verder. Dit ontwerp dient dan om de eigenlijke proof-of-concept te ontwikkelen. Tijdens de ontwikkeling wordt het bouwproces gedocumenteerd alsook mogelijke wijzigingen aan het ontwerp of impedimenten die verschijnen. Vervolgens wordt het systeem gevalideerd door de verwachte input te simuleren die vanuit een welgekozen proces zou komen om te zien of het voldoet aan de vereisten. \newline

Het laatste luik bestaat uit een grondige analyse van de resultaten waarin wordt nagegaan of het product voldoet en waar nog verdere verbeteringen mogelijk zijn. Op basis van deze resultaten kunnen we conclusies trekken en enige openstaande vragen binnen de twee domeinen beantwoorden.\newline
 
Vanuit deze aanpak wordt een grondige synthese opgebouwd waarin niet enkel de onderzoeksvraag en deelvragen worden beantwoord, maar tevens ook tot nieuwe inzichten die breed toepasbaar zijn voor zowel het casusbedrijf als het breder werkveld.


