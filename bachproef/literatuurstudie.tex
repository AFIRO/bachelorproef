%%=============================================================================
%% Literatuurstudie
%%=============================================================================

\chapter{\IfLanguageName{dutch}{Literatuurstudie}{Corpus Research}}%
\label{ch:literatuurstudie}
\section{\IfLanguageName{dutch}{Processen en Governance}{Processen en Governance}}%
\label{sec:processen en governance}
Een bedrijf of een onderneming heeft als hoofdtaak meerwaarde te creëren voor zichzelf, zijn klanten of de maatschappij. Dit is echter niet iets dat zo maar kan gebeuren. De mensen en systemen die daar werken moeten immers een aantal handelingen stellen om te komen tot deze meerwaarde. Wanneer dit de vorm aanneemt van een reeks gestructureerde handelingen die op basis van een startpunt, een zogenaamde trigger, begint, dezelfde flow herhaaldelijk volgt en uitkomt tot een specifiek eindresultaat, dan spreken we van een proces. Een eindresultaat van een proces kan dan ook de trigger zijn voor een ander proces. Hierbij zal het bedrijf vaak meerdere inputs, zoals grondstoffen en zijn eigen kennis of ervaring, gebruiken om een output te generen die meerwaarde creëert. Een samenvattende definitie luidt als volgt:
\begin{center}
\textit{“Een proces is een reeks gecoördineerde handelingen waarbij actoren op basis van een trigger, gebruikmakend van meerdere inputs, een output generen waarbij er meerwaarde wordt gecreëerd.”}
\end{center}\newline

Deze meerwaarde kan allerlei vormen aannemen. Voorbeelden zijn een product, een dienst, een opgelost probleem, specifieke informatie die de vrager nodig heeft, etc. Het creëren van deze meerwaarde is de bestaansreden van een bedrijf en zal er idealiter voor zorgen dat er genoeg inkomsten verdiend worden om het voortbestaan van het bedrijf te garanderen.\autocite[p. 4]{Dumas2018} \autocite[p. 5]{Weske2019} \newline

Een proces wordt opgebouwd uit veelal dezelfde bouwstenen: \autocite[pp. 3-4]{Dumas2018}
\begin{itemize}
  \item Actoren: dit zijn de mensen, maar ook organisaties en systemen die handelingen uitvoeren binnen het proces. Deze kunnen zowel intern als extern zijn. Een belangrijke actor is bijvoorbeeld de klant.
  \item Activiteit: iets dat moet gedaan worden binnen het proces. Als de activiteit complex genoeg is, kan dit aanleiding geven tot een eigen sub proces. 
  \item Event: een atomair gebeurtenis binnen het proces waar wij rekening mee moeten houden en die kan dienen als trigger of doel.
  \item Beslissingspunten: momenten binnen een proces waar deze kan aftakken op basis van specifieke criteria.
  \item Fysieke objecten: dit betreft alle wezenlijk bestaande objecten die nodig zijn om het proces tot een goed einde te brengen. Voorbeelden zijn grondstoffen, machines en gereedschap.
  \item Informatie objecten: alle objecten die vooral gelinkt zijn aan kennis. Voorbeelden zijn handleidingen, procedures of een boekhouding.
  \item Afloop: dit is het eindpunt van het proces waarbij potentieel meerwaarde wordt gerealiseerd. Een proces kan immers meerdere aflopen hebben, zowel positief als negatief.
\end{itemize}

Een bedrijf streeft om zijn processen zodanig te ontwerpen opdat deze zowel efficiënt als kwalitatief vol zijn. Een efficiënt proces zal immers rendabeler zijn qua tijd en kosten, terwijl een kwalitatief proces zal leiden tot kwalitatief product, wat het aanzien en de reputatie van het bedrijf ten goede komt. \autocite[p. 2]{Dumas2018} \newline

Echter, een sterk proces is enkel maar zo sterk als zijn uitvoering. Om de uitvoering van een proces te verbeteren, streven bedrijven naar een strakke governance of beheersing. Hierbij is het einddoel dat alle processen binnen een bedrijf niet enkel gekend en geoptimaliseerd zijn, maar dat deze processen door alle actoren op exact dezelfde manier, in exact dezelfde volgorde en met exact dezelfde meetbare criteria voor succes worden uitgevoerd onder het toezicht van een orgaan van proces owners, experten en managers die de kwaliteit hiervan garanderen en continu verbeteren. \autocite[p. 79]{Harmon2014} \newline

De voordelen van dergelijke aanpak zijn dat het alle meetbare key performance indicatoren ten goede komt, het een consistent eindproduct garandeert die voldoet aan compliance voorschriften en ownership van het proces plaatst bij een aantal sleutelfiguren die voor heel het bedrijf streven naar optimalisatie. Voor de medewerkers zelf zorgt dit voor een duidelijke workflow en de mogelijkheid om hun werk te plannen. Tevens verwijdert het ook het gevoel van ambiguïteit in hun werk. \autocite[p. 7]{Braganza2000} \newline

Dit is echter geen sinecure. Voor het uitwerken van een correcte governance structuur zijn een groot aantal frameworks beschikbaar, maar hierbij kan IT zeker en vast een meerwaarde leveren door de uitvoering van deze processen na te kijken, de juiste data te voorzien aan de decision makers en de medewerkers te assisteren in het uitvoeren van het proces. \newline

\section{\IfLanguageName{dutch}{Proces Monitoring}{Proces Monitoring}}%
\label{sec:proces monitoring}
Proces monitoring is, zoals de naam wel doet vermoeden, elk mogelijk systeem dat de uitvoering van een proces observeert en daaruit relevante data haalt opdat we tot nieuwe inzichten en verbetering kunnen komen. Het idee hier is dat data, vaak in de vorm van complexe event logs, gejuxtaposeerd wordt tegenover de key performance indicatoren die werden vastgelegd toen het proces werd gemodelleerd. Dit dient enerzijds om te ontdekken of de uitvoering voldoet aan de bedrijfsvisie, anderzijds waar de uitvoering hapert en uiteindelijk hoe het verder kan verbeterd worden. \autocite[pp. 413-414]{Dumas2018} \newline

Binnen het domein van proces monitoring bestaan er vier types. Offline proces monitoring bekijkt de historische uitvoeringsdata van het proces over een bepaalde periode in het verleden. Deze historische analyse is vooral nuttig als er veel data voor handen is opdat trends en de invloed van allerhande externe factoren kunnen vergeleken worden met elkaar. Het doel van deze monitoring is dan vooral het proces verbeteren op basis van historische data. \newline

De tegenhanger hiervan is online proces monitoring waarbij de huidig lopende uitvoeringen van het proces wordt nagekeken. Door gebruik te maken van near-real-time monitoring technieken kan er ingegrepen worden op het lopend proces wanneer bepaalde key performance indicatoren of service-level agreements overschreden zijn door alarmbellen of remediërende stappen te triggeren voordat verdere processen of de klant hier iets van merken. Een schoolvoorbeeld hiervan is een monitoring die de doorlooptijd van alle lopende dossiers bijhoudt en de medewerkers of de bevoegde leidinggevende contacteert als een dossier dreigt over termijn te gaan. \newline

Een verder onderscheid kan gemaakt worden tussen statistics-based en model-based technieken. De eerste gebruikt de ruwe data van de event logs om algemene statistische gegevens te berekenen rond de uitvoering van elke activiteit zoals de gemiddelde doorlooptijd, de deviatie van dit gemiddelde, maximale doorlooptijd en minimale doorlooptijd. Dit opent dan de deur voor makkelijke visualisatie in de vorm van operationele dashboards en vormt een subset van proces monitoring genaamd business activity monitoring. De andere legt de ruwe data naast het procesmodel en visualiseert de tijd en doorlooptijd van verschillende delen van het proces om zo deviaties te ontdekken tussen het ideale en het actuele procesverloop. Deze verregaande analyse, die als doel heeft om zowel het ideale als het actuele verloop tot in het uiterste te optimaliseren, wordt proces mijnen genoemd. \autocite[pp. 414-415]{Dumas2018}\newline

Hoewel proces monitoring zeer duidelijke meerwaarde biedt voor governance verantwoordelijken, analisten en owners, is het ook zeer moeilijk om op te zetten. In een omgeving met een hoge maturiteit inzake zowel proces governance als IT kan er monitoring ingebouwd worden in de IT-systemen van het bedrijf om de handelingen en de doorlooptijd van deze handelingen te capteren. De realisatie hiervan hangt echter af van zowel systemen als de materie. IT-systemen van eigen makelij kunnen aangepast worden om deze monitoring te doen, maar gekochte software zal deze optie niet altijd aanbieden en is vaak niet zomaar aan te passen door de klant. Daarnaast kunnen bepaalde stappen van het proces privacygevoelig zijn waardoor een legale analyse mogelijk nodig is voor dit kan geïmplementeerd worden om het bedrijf niet bloot te stellen aan risico’s. De rechten van de werknemer rond privacy op het werk zijn hierbij ook een twistpunt. Een laatste valkuil zijn activiteiten binnen het proces die niet plaatsvinden binnen de IT-systemen, zoals het invullen van een papieren formulier, of activiteiten die uitgevoerd worden door actoren volledig buiten het bedrijf. Hierbij moet onderzocht worden welke integratiepunten er gevonden kunnen worden als meetpunten en moeten de key performance metrieken aangepast worden om hiermee rekening te houden.\newline

\section{\IfLanguageName{dutch}{Proces Orkestratie}{Proces Orkestratie}}%
\label{sec:proces orkestratie}
Het bestaan van deze proces monitoring data laat onze toe om processen te automatiseren, wat van groot interesse is voor bedrijven gezien hier enorm veel tijdwinst en efficiëntiewinst geboekt kan worden mits de nodige investeringen. Proces automation stelt dat elk stuk van een proces dat uit handen genomen kan worden van een menselijke actor en vervangen kan worden door een IT-systeem geautomatiseerd dient te worden. De redenering hier is logisch. Een computersysteem zal specifieke delen van het proces vaak sneller en goedkoper uitvoeren dan zijn menselijke tegenhanger, waardoor deze zich kan buigen over de cruciale taken die niet te automatiseren zijn. Dit kan gaan van simpele taken tot zelfs volledige processen zoals het onderhouden en versturen van de stock van een volledig warenhuis. Dit is iets wat momenteel opkomt bij bedrijven zoals Amazon.  Dergelijke systemen zijn dan ook dagelijkse kost bij bedrijven in de vorm van ERP, CRM, SCM en PLM-pakketten. \autocite[pp. 341-343]{Dumas2018}\newline

Proces orkestratie gaat een stap verder dan automatisatie en koppelt het aan een procesmodel om zo de uitvoerders van deze processen te ondersteunen. Het resultaat is dan een business procesmanagement systeem dat niet enkel het proces kent, maar ook kan dirigeren. Door alle lopende procesinstanties op te volgen, weet het aan welke stappen de relevante actoren zitten en kan het intelligente suggesties maken voor de volgende taak van elke medewerker. Zodoende kan het systeem bottlenecks voorkomen en optimalisatie garanderen door elke medewerker de juiste taak op het juiste moment te laten uitvoeren. \newline

Dergelijke systemen kunnen op verschillende manieren worden opgebouwd, maar centraal hierbij is de executie engine. Dit is het brein van het systeem waarin alle logica wordt uitgevoerd en alle connecties gemaakt worden. Deze houdt alle lopende processen bij op basis van de proces monitoring data en legt deze naast het procesmodel om te weten aan welke stap het proces zit volgens het model. Op basis van berekeningen worden externe IT-services aangesproken om hun stuk van het verhaal te doen zodra dit nodig is. Daarnaast stuurt het de medewerkers aan door hen een takenlijst aan te bieden en hun vooruitgang op te volgen zodat ze altijd de handeling uitvoeren die het meest efficiënt is op dat moment. Vaak gaan deze systemen ook tools bevatten waarin de processen gemodelleerd kunnen worden opdat de executie engine deze kan begrijpen of kunnen ze inhaken op een externe databasis aan procesmodellen. \autocite[pp. 344-351]{Dumas2018} \newline

De voordelen van dergelijk systeem eindigen niet bij louter automatisatie. Het systeem kan ook integreren met de verschillende softwarepakketten binnen het bedrijf om hen aan te sturen. Hierdoor is het een waardevol integratiepunt voor een bedrijf dat gespecialiseerde software gebruikt in zijn verschillende afdelingen. Het dwingt verder compliance aan het proces af door medewerkers te verplichten bepaalde handelingen in een bepaalde volgorde uit te voeren. Als laatste voordeel maakt het de uitvoering van het proces veel transparanter omdat het zijn eigen handelingen en instructies kan loggen. Deze kunnen dan naast de uitvoeringslogs van de externe systemen en de medewerkers gelegd worden om te zien waar er mogelijke bottlenecks zijn. \newline

Uitdagingen en valkuilen bij dergelijke systemen zijn er uiteraard ook. De complexiteit bij het implementeren van dergelijk systeem is hoog en onderhoud is een continu proces, zeker als het de bedoeling is dat processen iteratief worden bijgestuurd. Daarnaast is het integreren met externe systemen niet altijd evident. Indien het extern systeem zelfs gestuurd kan worden door een orkestratiesysteem, dan is dit extern systeem zich niet bewust van het totaal proces of zijn plaats hierin. Vendors van externe systemen bieden vaak integraties aan met andere bekende systemen, maar dit is veelal duur en maakt het IT-systeem van het bedrijf meer een zwarte doos voor het intern team met hoog risico van vendor lock-in. Een andere uitdaging is menselijk. Het invoeren van een orkestratie systeem kan stuiten op weerstand omdat het werk daardoor niet enkel geroutineerder wordt, maar het veelal problemen blootlegt die men op allerlei niveaus van de onderneming liever zou negeren. Een sterk change management traject is vaak nodig om deze valkuilen te overbruggen. \autocite[pp. 360-365]{Dumas2018} \newline

\section{\IfLanguageName{dutch}{Proces Monitoring en Orkestratie als IT-systeem}{Proces Monitoring en Orkestratie als IT-systeem}}%
\label{sec:proces monitoring en orkestratie als IT-systeem}
Zoals hiervoor aangehaald is de theoretische architectuur heel duidelijk en rechtlijnig. Een executie engine bevraagt de monitoring voor data rond de courante proces instanties. Door deze naast het procesmodel te leggen, kan de engine gefundeerd zowel medewerkers als externe systemen aansturen door middel van respectievelijk een takenlijst en API-calls. Er zijn veel opties om dergelijk systeem te bouwen, maar algemeen bekeken zijn er twee courante strekkingen op de markt. \newline

Enerzijds is er de monoliet aanpak die gevolgd wordt door bijvoorbeeld IBM’s Business Process Manager product of de producten van Bizagi. Hierbij zijn de monitoring tools, de procesmodel tool, de executie engine en de takenlijst allemaal verbonden aan elkaar als deel van een groot softwarepakket dat draait op dezelfde server. De proces analisten bouwen in de tool de afgesproken processen na en de nodige front-end façades worden gebouwd zodat alle actoren rechtstreeks kunnen werken op het systeem en hun aansturing kunnen krijgen. Door middel van API-calls kan het systeem dan de uitgaande requests en binnenkomende responses nakijken om zicht te krijgen op de activiteiten van het proces die buiten zijn eigen systeem vallen. \newline

Dit soort design is typisch voor oudere enterprise pakketten en vindt zijn inspiratie dan ook bij de lange geschiedenis van ERP, CRM, SCM en PLM-pakketten. Hoewel dit voor moderne softwareontwikkelaars oubollig kan lijken, zijn de voordelen niet te onderschatten. Doordat de databasis voor zowel monitoring, model als executie engine dezelfde is, kunnen er sterke optimalisaties gedaan worden op databasis managementsysteem niveau zodat dit geen vertragende factor kan zijn. Verder is het feit dat alles een geïntegreerde monoliet is ook zeer nuttig om een lage latency en snelle dataverwerking te garanderen. Gezien de input van de medewerkers ook rechtstreeks in het systeem gebeurt, kan de engine dit ook veel sneller verwerken. \newline

Deze aanpak heeft echter ook nadelen. Dergelijke systemen kunnen moeilijk horizontaal schalen gezien extra instanties starten om de werklast te balanceren nutteloos is. Men zou al processen moeten verspreiden over verschillende systemen die elk verantwoordelijk zijn voor hun eigen domein om een soortgelijk effect te bekomen. Verticaal schalen is uiteraard een optie, maar die is qua kostprijs niet ideaal. Verder is het uitbreiden van dergelijke systemen niet makkelijk. Als het in-house gebouwd systeem is, dan zal elke uitbreiding bijdragen tot de groeiende complexiteit die gekend is bij monolieten en indien het aangekocht is, dan zijn extensies altijd duur. De laatste valkuil hierbij is dan ook vendor lock-in. Na de setup van dergelijk systeem is een migratie vaak een zeer duur en moeizaam project. \autocite[p. 348]{Dumas2018} \newline

Een andere en modernere aanpak is de gedistribueerde aanpak op basis van hedendaagse microservices architectuur met eventing. In dit model vormen de proces monitoring tool en de proces orkestratie executie engine hun eigen microservice met gelinkte databasis. Elke front-end waar een eindgebruiker in werkt en back-end service die een rol speelt in het proces krijgt dan een library die hen toelaat om events te genereren en om die via een event broker op een queue te plaatsen. De proces monitor tool zal deze events dan kunnen consumeren om zo analoog aan de event logging van een monolithisch systeem een beeld te krijgen van waar de verschillende procesinstanties zich op dit moment bevinden en om deze data bij te houden voor zowel online als offline proces monitoring. De proces orkestratie tool zal dan analoog aan de executie engine zowel de proces monitoring tool als een proces documentatie tool bevragen die als bron van waarheid fungeert voor het hele proces. Op basis van deze input kan deze dan zowel externe services als medewerkers aansturen door via API-calls de nodige requests te sturen naar beiden. \newline

De voordelen van deze aanpak zijn uiteraard gelijkaardig aan de algemene voordelen van microservices architectuur met eventing die zeer standaard is in hedendaagse softwareontwikkeling. Door kleine services met specifieke verantwoordelijkheden te gebruiken verhogen we onze schaalbaarheid via horizontale werklast balanceren door extra instanties te starten die naar dezelfde databasis schrijven. Onderhoudbaarheid is dan ook hoger omdat de services kleiner zijn dan een grote monoliet, waardoor deze door meerdere teams onderhouden kunnen worden met snellere verbeter en release cycli. Door gebruik van eventing is deze architectuur ook veel robuuster. Er kan immers een microservice falen zonder impact te hebben op de rest van de architectuur en met minimaal dataverlies omdat events buiten de microservice bewaard worden in de event queue van de broker. Een specifiek voordeel aan deze aanpak is dat de proces monitor en proces orkestratie tool niet meer hard aan elkaar gekoppeld zijn. Deze ontkoppeling geeft ons meer opties om beide services bloot te stellen als API’s voor mogelijke afnemers voor bijvoorbeeld business intelligence rapportage of data science analyse. We zijn niet meer gebonden aan de façades van het BPMS-systeem en kunnen het systeem veel makkelijker integreren in onze eigen front-ends en back-ends om zo winsten te behalen op vlak van UI/UX-design en performante moderne frameworks. Een laatste voordeel is dat we de optie behouden om naast onze eigen tools ook out-of-the-box tools in deze opstelling op te nemen zonder in te boeten aan flexibiliteit. Als blijkt dat het nuttiger is om bijvoorbeeld de proces monitor of procesmodel tool te kopen, dan kan dit gerust mits deze een flexibele API aanbiedt die voldoet aan onze vereisten qua data. \newline

Qua nadelen verliezen we vooral de snelheid en low-latency van een monoliet op een on-premise server. Ze gaan immers geen deel meer uitmaken van hetzelfde systeem, maar onrechtstreeks met elkaar communiceren via events of rechtstreeks via API-calls. Afhankelijk van hoe dit systeem is opgezet zal elke microservice veelal in eigen containers of clusters draaien on-premise of in de cloud. Dit zal logischerwijs zorgen voor een hogere latency. Wat we winnen aan flexibiliteit en onderhoudbaarheid, verliezen we in complexere architectuur waarbij een probleem end-2-end traceren moeilijker wordt. \autocite[pp. 8-22]{Janiesch2012} \newline

De vraag is dan ook welk van deze twee aanpakken het beste past bij de specifieke use-cases van onze onderzoeksvraag. Beiden zouden werken, maar gelet op de specifieke situatie van het casusbedrijf, lijkt de microservice architectuur een logischere keus. Het casusbedrijf heeft immers veel custom development binnen zijn gebruikte software waar makkelijker op kan worden ingehaakt net omdat de code kan worden aangepast om events te versturen als reactie op specifieke activiteiten die het procesmodel volgen. Voor software die niet in-house werd gemaakt, bestaat oftewel de optie om die te configureren om REST-calls te sturen na specifieke activiteiten of is er een custom integratie naar de eigen systemen die dit kan doen. Een laatste belangrijke consideratie is dat het casusbedrijf intern draait op een microservices architectuur met eventing, waardoor dit systeem volledig zou passen binnen het architecturaal model en zou kunnen inhaken op de bestaande infrastructuur rond eventing en containerisatie in de cloud. Verdere analyse rond requirements en technische vereisten zal echter uitwijzen of dit daadwerkelijk de meest gepaste aanpak \newline
