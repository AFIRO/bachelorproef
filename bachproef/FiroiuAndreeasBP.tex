%===============================================================================
% LaTeX sjabloon voor de bachelorproef toegepaste informatica aan HOGENT
% Meer info op https://github.com/HoGentTIN/latex-hogent-report
%===============================================================================

\documentclass[dutch,dit,thesis]{hogentreport}

\usepackage{lipsum} % For blind text, can be removed after adding actual content

%% Pictures to include in the text can be put in the graphics/ folder
\graphicspath{{../graphics/}}

%% For source code highlighting, requires pygments to be installed
%% Compile with the -shell-escape flag!
%% \usepackage[chapter]{minted}
%% If you compile with the make_thesis.{bat,sh} script, use the following
%% import instead:
\usepackage[chapter,outputdir=../output]{minted}
\usemintedstyle{solarized-light}

%% Formatting for minted environments.
\setminted{%
    autogobble,
    frame=lines,
    breaklines,
    linenos,
    tabsize=4
}

%% Ensure the list of listings is in the table of contents
\renewcommand\listoflistingscaption{%
    \IfLanguageName{dutch}{Lijst van codefragmenten}{List of listings}
}
\renewcommand\listingscaption{%
    \IfLanguageName{dutch}{Codefragment}{Listing}
}
\renewcommand*\listoflistings{%
    \cleardoublepage\phantomsection\addcontentsline{toc}{chapter}{\listoflistingscaption}%
    \listof{listing}{\listoflistingscaption}%
}

% Other packages not already included can be imported here

%%---------- Document metadata -------------------------------------------------
% TODO: Replace this with your own information
\author{Andreeas Firoiu}
\supervisor{Dhr. M. Asselberg}
\cosupervisor{Dhr. R. Van Limbergen}
\title 
    {Requirements analyse en proof of concept van een raamwerk voor custom process monitoring en orchestration}
\academicyear{\advance\year by -1 \the\year--\advance\year by 1 \the\year}
\examperiod{1}
\degreesought{\IfLanguageName{dutch}{Professionele bachelor in de toegepaste informatica}{Bachelor of applied computer science}}
\partialthesis{false} %% To display 'in partial fulfilment'
%\institution{Internshipcompany BVBA.}

%% Add global exceptions to the hyphenation here
\hyphenation{back-slash}

%% The bibliography (style and settings are  found in hogentthesis.cls)
\addbibresource{bachproef.bib}            %% Bibliography file
\addbibresource{../voorstel/voorstel.bib} %% Bibliography research proposal
\defbibheading{bibempty}{}

%% Prevent empty pages for right-handed chapter starts in twoside mode
\renewcommand{\cleardoublepage}{\clearpage}

\renewcommand{\arraystretch}{1.2}

%% Content starts here.
\begin{document}

%---------- Front matter -------------------------------------------------------

\frontmatter

\hypersetup{pageanchor=false} %% Disable page numbering references
%% Render a Dutch outer title page if the main language is English
\IfLanguageName{english}{%
    %% If necessary, information can be changed here
    \degreesought{Professionele Bachelor toegepaste informatica}%
    \begin{otherlanguage}{dutch}%
       \maketitle%
    \end{otherlanguage}%
}{}

%% Generates title page content
\maketitle
\hypersetup{pageanchor=true}

%%=============================================================================
%% Voorwoord
%%=============================================================================

\chapter*{\IfLanguageName{dutch}{Woord vooraf}{Preface}}%
\label{ch:voorwoord}

Tijdens mijn werk als Fullstack DevOps en Functioneel & Business Analist ben ik bij mijn werkgever Liantis in contact komen met een groot aantal problemen die zowel interessant als hoofdbrekend waren om op te lossen. Toen er intern interesse bleek voor het proces governance en later proces monitoring en orkestratie vond ik het onderwerp boeiend, maar de kansen waren er niet om het verder uit te werken. Toch bleef dit in mijn achterhoofd spoken als interessant onderwerp voor verder onderzoek. De bachelorproef was dan ook een uitstekende gelegenheid om dit verder uit te werken in de hoop hier intern iets verder mee te kunnen doen. De business value voor het bedrijf was zeker aanwezig en het onderwerp bood daardoor ook interessante professionele kansen voor mij. 

Sinds de start van dit onderzoek heb ik veel geleerd over het onderwerp en zijn er concrete stappen ondernomen om proces monitoring en orkestratie om te zetten van een theoretische oefening naar een bestaand product. Ik ben dan ook enorm tevreden dat ik hierin heb kunnen bijdragen en mijn stempel kon drukken op een bedrijfskritisch systeem van mijn werkgever.

Ik wil hierbij dan ook een aantal mensen bedanken. Allereerst mijn promotor Marc Asselberg wiens feedback en ervaring onontbeerlijk was bij het schrijven van dit werk en mijn co-promotor Robin Van Limbergen wiens hands-on ervaring in allerlei domeinen binnen het werkveld zeer inspirerend werkte. Ook mijn echtgenote Nathalie De Baere die mij in dit vijfjarig afstandsleren avontuur dagdagelijks heeft gesteund verdient alle lof voor haar geduld en tedere woorden. Verder wil ik mijn collega's bij Liantis uit Team Vetstrak en Team Hyperloop bedanken om een warme omgeving te bieden waarin een junior met veertig geslaagde studiepunten kon uitgroeien tot een sterke IT-professional. Ik wil tevens de collega's van Team Nexus goede moed wensen bij voorbaat bij het uitwerken van het systeem dat ik dit tijdens dit werk heb uitgedacht.    

\lipsum[1-2]
\IfLanguageName{english}{%
\selectlanguage{dutch}
\chapter*{Samenvatting}

\selectlanguage{english}
}{}


\chapter*{\IfLanguageName{dutch}{Samenvatting}{Abstract}}

Dit werk onderzoekt hoe we proces monitoring en orkestratie kunnen implementeren binnen een Belgisch bedrijf met een IT-context waarbij er veel legacy toepassingen zijn. De centrale onderzoeksvraag is: \textbf{”Wat zijn de vereisten waaraan een systeem moet voldoen en welke software architectuur is gepast om custom proces monitoring en orkestratie te implementeren binnen een bedrijf met veel custom development applicaties?”}.\newline

Dit onderzoek beoogt een praktisch raamwerk te ontwikkelen met een proof-of-concept dat de toepassing van deze technologieën in een governance-structuur beschrijft en faciliteert. Verder wordt dit raamwerk praktisch toegepast op het bedrijf Liantis met een eerste proof-of-concept waarin een specifiek proces wordt ondersteund met monitoring en orkestratie. Theoretisch en marktonderzoek wordt gebruikt om te ontdekken wat er gangbaar is binnen het domein, waarnaar we dit toetsen aan de noden van het casusbedrijf voor een praktische requirements analyse en proof-of-concept. Voor de proof-of-concept wordt vervolgens een simulatie van het test-proces uitgevoerd om zo een waarheidsgetrouw resultaat te leveren waaruit wij verdere conclusies kunnen trekken rond de praktische toepasbaarheid van het systeem en verdere iteratieve verbeteringen.


%---------- Inhoud, lijst figuren, ... -----------------------------------------

\tableofcontents

% In a list of figures, the complete caption will be included. To prevent this,
% ALWAYS add a short description in the caption!
%
%  \caption[short description]{elaborate description}
%
% If you do, only the short description will be used in the list of figures

\listoffigures

% If you included tables and/or source code listings, uncomment the appropriate
% lines.

\listoflistings

% Als je een lijst van afkortingen of termen wil toevoegen, dan hoort die
% hier thuis. Gebruik bijvoorbeeld de ``glossaries'' package.
% https://www.overleaf.com/learn/latex/Glossaries

%---------- Kern ---------------------------------------------------------------

\mainmatter{}

% De eerste hoofdstukken van een bachelorproef zijn meestal een inleiding op
% het onderwerp, literatuurstudie en verantwoording methodologie.
% Aarzel niet om een meer beschrijvende titel aan deze hoofdstukken te geven of
% om bijvoorbeeld de inleiding en/of stand van zaken over meerdere hoofdstukken
% te verspreiden!

%%=============================================================================
%% Inleiding
%%=============================================================================

\chapter{\IfLanguageName{dutch}{Inleiding}{Introduction}}%
\label{ch:inleiding}
\section{\IfLanguageName{dutch}{Probleemstelling}{Problem Statement}}%
\label{sec:probleemstelling}

Bedrijven streven constant naar verbetering en optimalisatie. Dit is een verhaal van alle tijden, maar in het huidige competitieve bedrijfsleven is er weinig plaats voor inefficiënte en verspilling. Nu markten geglobaliseerd zijn en de drempel voor concurrentie lager dan ooit is, zijn het de bedrijven die hun processen tot in de puntjes beheersen en automatiseren die de beste resultaten behalen.\newline

Bij Liantis, een middelgroot HR-bedrijf, is deze beheersing vol op aan de gang. Intern heerst al lang de visie dat onze grootste troef schuilt in krachtige dienstverlening ondersteund door sterke IT. Process governance is op alle niveaus van de onderneming ingevoerd. De flows zijn geanalyseerd en de bottlenecks zoveel mogelijk weggewerkt. Er is echter nog veel potentieel voor grotere winsten in efficiëntie. We weten wat “de juiste manier” is om een proces uit te voeren en medewerkers volgen dit zo goed mogelijk, maar tussen start en stop is er een zwart gat aan informatie die in deze tijden van grote data een gemiste kans is voor captatie, rapportage en verbetering via automatisatie.\newline 

Deze nood bij Liantis was dan ook de aanleiding voor dit werk. Er is duidelijk plaats in de onderneming voor systemen die de processen monitoren en automatiseren. De winsten zouden dan ook legio zijn, maar intern missen we kennis en ervaring om hier iets mee te doen. Dergelijk project uit het niets starten en buy-in krijgen van het management is geen sinecure Een bijkomend einddoel van dit onderzoek is dan ook om het resultaat van deze bachelor proef te gebruiken als ondersteuning voor een projectvoorstel met proof-of-concept voor een formele implementatie van een dergelijk systeem binnen de onderneming.
\section{\IfLanguageName{dutch}{Onderzoeksvraag}{Research question}}%
\label{sec:onderzoeksvraag}

De centrale onderzoeksvraag is dan ook opgebouwd vanuit deze optiek. \textbf{”Wat zijn de vereisten waaraan een systeem moet voldoen en welke software architectuur is gepast om custom proces monitoring en orkestratie te implementeren binnen een bedrijf met veel custom development toepassingen?”}. Hierbij is het adjectief custom zeer cruciaal. Er zijn immers genoeg out-of-box oplossingen die kunnen inhaken op bestaande software om dit probleem op te lossen. Echter, net zoals veel Belgische bedrijven, is Liantis organisch gegroeid en is gehanteerde software van eigen makelij. Dit zorgt dan ook vaak voor moeilijkheden als bestaande software wordt aangekocht omdat custom integraties geschreven moeten worden. Een systeem dat kan inhaken op de specifieke noden qua processen en software van het bedrijf zal dus evenzeer van eigen makelij moeten zijn.\newline

Om deze onderzoeksvraag correct te beantwoorden, wordt zowel het probleem als het oplossingsdomein opgesplitst in deelvragen als volgt.

\section{\IfLanguageName{dutch}{Probleemdomein}{Probleemdomein}}%
\label{sec:probleemdomein}
\begin{itemize}
  \item Wat zijn de functionele en niet-functionele vereisten waaraan dit systeem moet voldoen?
  \item Wat zijn de mogelijkheden die bestaande proces monitoring en orkestratie software op de markt bieden?
  \item Wat zijn de specifieke functionele en non-functionele vereisten qua proces monitoring en orkestratie voor processen binnen een HR-bedrijf?
  \item Welke technische en architecturale uitdagingen zijn er bij de implementatie van dergelijke systemen in een bedrijfsomgeving met veel custom development?
\end{itemize}

\section{\IfLanguageName{dutch}{Oplossingsdomein}{Probleemdomein}}%
\label{sec:oplossingsdomein}
\begin{itemize}
  \item Hoe implementeren we dergelijke systemen in een bedrijfsomgeving met veel custom development?
  \item Aan welke criteria moet dergelijk system voldoen om succesvolle proces monitoring en orkestratie te bereiken binnen het casusbedrijf?
\end{itemize}

\section{\IfLanguageName{dutch}{Opzet van deze bachelorproef}{Structure of this bachelor thesis}}%
\label{sec:opzet-bachelorproef}

% Het is gebruikelijk aan het einde van de inleiding een overzicht te
% geven van de opbouw van de rest van de tekst. Deze sectie bevat al een aanzet
% die je kan aanvullen/aanpassen in functie van je eigen tekst.

De rest van deze bachelorproef is als volgt opgebouwd: \newline

In Hoofdstuk~\ref{ch:onderzoeksmethode} wordt de methodologie toegelicht en worden de gebruikte onderzoekstechnieken besproken om een antwoord te kunnen formuleren op de onderzoeksvragen.\newline

In Hoofdstuk~\ref{ch:literatuurstudie} wordt de huidige stand van zaken van het domein proces governance, proces monitoring en proces orkestratie binnen het werkveld om zo een theoretisch vertrekpunt te hebben voor dit werk.\newline

In Hoofdstuk~\ref{ch:analyse} wordt er enerzijds een requirementsanalyse uitgevoerd om de functionele noden van het casusbedrijf correct te capteren met mogelijke realisaties. Anderzijds wordt een technische en niet-functionele analyse uitgevoerd opdat het eindproduct past binnen het architecturaal kader van het bedrijf.\newline

In Hoofdstuk~\ref{ch:proof-of-concept} wordt op basis van de voorgaande theoretische achtergrond en de analyse zowel een theoretisch compleet als proof-of-concept systeem ontworpen. Verder wordt de proof-of-concept gebouwd en gevalideerd op basis van een simulatie van een bestaand proces van het casusbedrijf om te zien of het voldoet aan de requirementsanalyse.\newline

In Hoofdstuk~\ref{ch:conclusie}, tenslotte, worden de resultaten van de proof-of-concept besproken en wordt op basis hiervan gereflecteerd over de onderzoeksvraag voor een finale conclusie.
%%=============================================================================
%% Onderzoeksmethode
%%=============================================================================

\chapter{\IfLanguageName{dutch}{Onderzoeksmethode}{Methodology}}%
\label{ch:onderzoeksmethode}

Het onderzoek voor dit werk bestaat uit een vierledige aanpak waarin enerzijds de vragen binnen zowel het probleemdomein als het oplossingsdomein worden opgelost en anderzijds een proof-of-concept wordt onderzocht, ontworpen en gebouwd die voldoet aan functionele en non-functionele vereisten van het casusbedrijf Liantis. \newline

Allereerst wordt er een literatuurstudie gedaan waarin de nodige theoretische achtergrondkennis wordt verzameld om dit project tot een goed einde te brengen. Hierbij wordt eerst onderzocht hoe “process governance” werkt en hoe proces monitoring en orkestratie hierop inhaken om waarde te creëren voor het bedrijf door middel van toonaangevende werken binnen het domein. De meer technisch kant van het verhaal wordt verder onderzocht door de documentatie van reeds bestaande producten en proof-of-concepts te bekijken om te zien hoe deze opgebouwd zijn.\newline

De tweede fase bestaat uit een requirements en technische analyse om te ontdekken wat het casusbedrijf nodig heeft en wat zowel de functionele als niet-functionele vereisten zijn. Dit gebeurt enerzijds door middel van een workshop waarin de theoretische use-cases van een aantal stakeholders vanuit de business kant worden gecapteerd en anderzijds op basis van een sparring sessie met een systeem architect waarin wordt bekeken hoe dergelijk theoretisch systeem past binnen de architectuur van het bedrijf.\newline

De derde fase bestaat uit het ontwerpen en implementeren van de eigenlijke proof-of-concept. Op basis van de voorgenoemde vereisten ontwerpen we een systeem dat voldoet en documenteren we dit in de nodige deliverables zoals een architecturaal-diagram, een domeinmodel, een entiteit relatie diagram en verder. Dit ontwerp dient dan om de eigenlijke proof-of-concept te ontwikkelen. Tijdens de ontwikkeling wordt het bouwproces gedocumenteerd alsook mogelijke wijzigingen aan het ontwerp of impedimenten die verschijnen. Vervolgens wordt het systeem gevalideerd door de verwachte input te simuleren die vanuit een welgekozen proces zou komen om te zien of het voldoet aan de vereisten. \newline

Het laatste luik bestaat uit een grondige analyse van de resultaten waarin wordt nagegaan of het product voldoet en waar nog verdere verbeteringen mogelijk zijn. Op basis van deze resultaten kunnen we conclusies trekken en enige openstaande vragen binnen de twee domeinen beantwoorden.\newline
 
Vanuit deze aanpak wordt een grondige synthese opgebouwd waarin niet enkel de onderzoeksvraag en deelvragen worden beantwoord, maar tevens ook tot nieuwe inzichten die breed toepasbaar zijn voor zowel het casusbedrijf als het breder werkveld.



%%=============================================================================
%% Literatuurstudie
%%=============================================================================

\chapter{\IfLanguageName{dutch}{Literatuurstudie}{Corpus Research}}%
\label{ch:literatuurstudie}
\section{\IfLanguageName{dutch}{Processen en Governance}{Processen en Governance}}%
\label{sec:processen en governance}
Een bedrijf of een onderneming heeft als hoofdtaak meerwaarde te creëren voor zichzelf, zijn klanten of de maatschappij. Dit is echter niet iets dat zo maar kan gebeuren. De mensen en systemen die daar werken moeten immers een aantal handelingen stellen om te komen tot deze meerwaarde. Wanneer dit de vorm aanneemt van een reeks gestructureerde handelingen die op basis van een startpunt, een zogenaamde trigger, begint, dezelfde flow herhaaldelijk volgt en uitkomt tot een specifiek eindresultaat, dan spreken we van een proces. Een eindresultaat van een proces kan dan ook de trigger zijn voor een ander proces. Hierbij zal het bedrijf vaak meerdere inputs, zoals grondstoffen en zijn eigen kennis of ervaring, gebruiken om een output te generen die meerwaarde creëert. Een samenvattende definitie luidt als volgt:
\begin{center}
\textit{“Een proces is een reeks gecoördineerde handelingen waarbij actoren op basis van een trigger, gebruikmakend van meerdere inputs, een output generen waarbij er meerwaarde wordt gecreëerd.”}
\end{center}\newline

Deze meerwaarde kan allerlei vormen aannemen. Voorbeelden zijn een product, een dienst, een opgelost probleem, specifieke informatie die de vrager nodig heeft, etc. Het creëren van deze meerwaarde is de bestaansreden van een bedrijf en zal er idealiter voor zorgen dat er genoeg inkomsten verdiend worden om het voortbestaan van het bedrijf te garanderen.\autocite[p. 4]{Dumas2018} \autocite[p. 5]{Weske2019} \newline

Een proces wordt opgebouwd uit veelal dezelfde bouwstenen: \autocite[pp. 3-4]{Dumas2018}
\begin{itemize}
  \item Actoren: dit zijn de mensen, maar ook organisaties en systemen die handelingen uitvoeren binnen het proces. Deze kunnen zowel intern als extern zijn. Een belangrijke actor is bijvoorbeeld de klant.
  \item Activiteit: iets dat moet gedaan worden binnen het proces. Als de activiteit complex genoeg is, kan dit aanleiding geven tot een eigen sub proces. 
  \item Event: een atomair gebeurtenis binnen het proces waar wij rekening mee moeten houden en die kan dienen als trigger of doel.
  \item Beslissingspunten: momenten binnen een proces waar deze kan aftakken op basis van specifieke criteria.
  \item Fysieke objecten: dit betreft alle wezenlijk bestaande objecten die nodig zijn om het proces tot een goed einde te brengen. Voorbeelden zijn grondstoffen, machines en gereedschap.
  \item Informatie objecten: alle objecten die vooral gelinkt zijn aan kennis. Voorbeelden zijn handleidingen, procedures of een boekhouding.
  \item Afloop: dit is het eindpunt van het proces waarbij potentieel meerwaarde wordt gerealiseerd. Een proces kan immers meerdere aflopen hebben, zowel positief als negatief.
\end{itemize}

Een bedrijf streeft om zijn processen zodanig te ontwerpen opdat deze zowel efficiënt als kwalitatief vol zijn. Een efficiënt proces zal immers rendabeler zijn qua tijd en kosten, terwijl een kwalitatief proces zal leiden tot kwalitatief product, wat het aanzien en de reputatie van het bedrijf ten goede komt. \autocite[p. 2]{Dumas2018} \newline

Echter, een sterk proces is enkel maar zo sterk als zijn uitvoering. Om de uitvoering van een proces te verbeteren, streven bedrijven naar een strakke governance of beheersing. Hierbij is het einddoel dat alle processen binnen een bedrijf niet enkel gekend en geoptimaliseerd zijn, maar dat deze processen door alle actoren op exact dezelfde manier, in exact dezelfde volgorde en met exact dezelfde meetbare criteria voor succes worden uitgevoerd onder het toezicht van een orgaan van proces owners, experten en managers die de kwaliteit hiervan garanderen en continu verbeteren. \autocite[p. 79]{Harmon2014} \newline

De voordelen van dergelijke aanpak zijn dat het alle meetbare key performance indicatoren ten goede komt, het een consistent eindproduct garandeert die voldoet aan compliance voorschriften en ownership van het proces plaatst bij een aantal sleutelfiguren die voor heel het bedrijf streven naar optimalisatie. Voor de medewerkers zelf zorgt dit voor een duidelijke workflow en de mogelijkheid om hun werk te plannen. Tevens verwijdert het ook het gevoel van ambiguïteit in hun werk. \autocite[p. 7]{Braganza2000} \newline

Dit is echter geen sinecure. Voor het uitwerken van een correcte governance structuur zijn een groot aantal frameworks beschikbaar, maar hierbij kan IT zeker en vast een meerwaarde leveren door de uitvoering van deze processen na te kijken, de juiste data te voorzien aan de decision makers en de medewerkers te assisteren in het uitvoeren van het proces. \newline

\section{\IfLanguageName{dutch}{Proces Monitoring}{Proces Monitoring}}%
\label{sec:proces monitoring}
Proces monitoring is, zoals de naam wel doet vermoeden, elk mogelijk systeem dat de uitvoering van een proces observeert en daaruit relevante data haalt opdat we tot nieuwe inzichten en verbetering kunnen komen. Het idee hier is dat data, vaak in de vorm van complexe event logs, gejuxtaposeerd wordt tegenover de key performance indicatoren die werden vastgelegd toen het proces werd gemodelleerd. Dit dient enerzijds om te ontdekken of de uitvoering voldoet aan de bedrijfsvisie, anderzijds waar de uitvoering hapert en uiteindelijk hoe het verder kan verbeterd worden. \autocite[pp. 413-414]{Dumas2018} \newline

Binnen het domein van proces monitoring bestaan er vier types. Offline proces monitoring bekijkt de historische uitvoeringsdata van het proces over een bepaalde periode in het verleden. Deze historische analyse is vooral nuttig als er veel data voor handen is opdat trends en de invloed van allerhande externe factoren kunnen vergeleken worden met elkaar. Het doel van deze monitoring is dan vooral het proces verbeteren op basis van historische data. \newline

De tegenhanger hiervan is online proces monitoring waarbij de huidig lopende uitvoeringen van het proces wordt nagekeken. Door gebruik te maken van near-real-time monitoring technieken kan er ingegrepen worden op het lopend proces wanneer bepaalde key performance indicatoren of service-level agreements overschreden zijn door alarmbellen of remediërende stappen te triggeren voordat verdere processen of de klant hier iets van merken. Een schoolvoorbeeld hiervan is een monitoring die de doorlooptijd van alle lopende dossiers bijhoudt en de medewerkers of de bevoegde leidinggevende contacteert als een dossier dreigt over termijn te gaan. \newline

Een verder onderscheid kan gemaakt worden tussen statistics-based en model-based technieken. De eerste gebruikt de ruwe data van de event logs om algemene statistische gegevens te berekenen rond de uitvoering van elke activiteit zoals de gemiddelde doorlooptijd, de deviatie van dit gemiddelde, maximale doorlooptijd en minimale doorlooptijd. Dit opent dan de deur voor makkelijke visualisatie in de vorm van operationele dashboards en vormt een subset van proces monitoring genaamd business activity monitoring. De andere legt de ruwe data naast het procesmodel en visualiseert de tijd en doorlooptijd van verschillende delen van het proces om zo deviaties te ontdekken tussen het ideale en het actuele procesverloop. Deze verregaande analyse, die als doel heeft om zowel het ideale als het actuele verloop tot in het uiterste te optimaliseren, wordt proces mijnen genoemd. \autocite[pp. 414-415]{Dumas2018}\newline

Hoewel proces monitoring zeer duidelijke meerwaarde biedt voor governance verantwoordelijken, analisten en owners, is het ook zeer moeilijk om op te zetten. In een omgeving met een hoge maturiteit inzake zowel proces governance als IT kan er monitoring ingebouwd worden in de IT-systemen van het bedrijf om de handelingen en de doorlooptijd van deze handelingen te capteren. De realisatie hiervan hangt echter af van zowel systemen als de materie. IT-systemen van eigen makelij kunnen aangepast worden om deze monitoring te doen, maar gekochte software zal deze optie niet altijd aanbieden en is vaak niet zomaar aan te passen door de klant. Daarnaast kunnen bepaalde stappen van het proces privacygevoelig zijn waardoor een legale analyse mogelijk nodig is voor dit kan geïmplementeerd worden om het bedrijf niet bloot te stellen aan risico’s. De rechten van de werknemer rond privacy op het werk zijn hierbij ook een twistpunt. Een laatste valkuil zijn activiteiten binnen het proces die niet plaatsvinden binnen de IT-systemen, zoals het invullen van een papieren formulier, of activiteiten die uitgevoerd worden door actoren volledig buiten het bedrijf. Hierbij moet onderzocht worden welke integratiepunten er gevonden kunnen worden als meetpunten en moeten de key performance metrieken aangepast worden om hiermee rekening te houden.\newline

\section{\IfLanguageName{dutch}{Proces Orkestratie}{Proces Orkestratie}}%
\label{sec:proces orkestratie}
Het bestaan van deze proces monitoring data laat onze toe om processen te automatiseren, wat van groot interesse is voor bedrijven gezien hier enorm veel tijdwinst en efficiëntiewinst geboekt kan worden mits de nodige investeringen. Proces automation stelt dat elk stuk van een proces dat uit handen genomen kan worden van een menselijke actor en vervangen kan worden door een IT-systeem geautomatiseerd dient te worden. De redenering hier is logisch. Een computersysteem zal specifieke delen van het proces vaak sneller en goedkoper uitvoeren dan zijn menselijke tegenhanger, waardoor deze zich kan buigen over de cruciale taken die niet te automatiseren zijn. Dit kan gaan van simpele taken tot zelfs volledige processen zoals het onderhouden en versturen van de stock van een volledig warenhuis. Dit is iets wat momenteel opkomt bij bedrijven zoals Amazon.  Dergelijke systemen zijn dan ook dagelijkse kost bij bedrijven in de vorm van ERP, CRM, SCM en PLM-pakketten. \autocite[pp. 341-343]{Dumas2018}\newline

Proces orkestratie gaat een stap verder dan automatisatie en koppelt het aan een procesmodel om zo de uitvoerders van deze processen te ondersteunen. Het resultaat is dan een business procesmanagement systeem dat niet enkel het proces kent, maar ook kan dirigeren. Door alle lopende procesinstanties op te volgen, weet het aan welke stappen de relevante actoren zitten en kan het intelligente suggesties maken voor de volgende taak van elke medewerker. Zodoende kan het systeem bottlenecks voorkomen en optimalisatie garanderen door elke medewerker de juiste taak op het juiste moment te laten uitvoeren. \newline

Dergelijke systemen kunnen op verschillende manieren worden opgebouwd, maar centraal hierbij is de executie engine. Dit is het brein van het systeem waarin alle logica wordt uitgevoerd en alle connecties gemaakt worden. Deze houdt alle lopende processen bij op basis van de proces monitoring data en legt deze naast het procesmodel om te weten aan welke stap het proces zit volgens het model. Op basis van berekeningen worden externe IT-services aangesproken om hun stuk van het verhaal te doen zodra dit nodig is. Daarnaast stuurt het de medewerkers aan door hen een takenlijst aan te bieden en hun vooruitgang op te volgen zodat ze altijd de handeling uitvoeren die het meest efficiënt is op dat moment. Vaak gaan deze systemen ook tools bevatten waarin de processen gemodelleerd kunnen worden opdat de executie engine deze kan begrijpen of kunnen ze inhaken op een externe databasis aan procesmodellen. \autocite[pp. 344-351]{Dumas2018} \newline

De voordelen van dergelijk systeem eindigen niet bij louter automatisatie. Het systeem kan ook integreren met de verschillende softwarepakketten binnen het bedrijf om hen aan te sturen. Hierdoor is het een waardevol integratiepunt voor een bedrijf dat gespecialiseerde software gebruikt in zijn verschillende afdelingen. Het dwingt verder compliance aan het proces af door medewerkers te verplichten bepaalde handelingen in een bepaalde volgorde uit te voeren. Als laatste voordeel maakt het de uitvoering van het proces veel transparanter omdat het zijn eigen handelingen en instructies kan loggen. Deze kunnen dan naast de uitvoeringslogs van de externe systemen en de medewerkers gelegd worden om te zien waar er mogelijke bottlenecks zijn. \newline

Uitdagingen en valkuilen bij dergelijke systemen zijn er uiteraard ook. De complexiteit bij het implementeren van dergelijk systeem is hoog en onderhoud is een continu proces, zeker als het de bedoeling is dat processen iteratief worden bijgestuurd. Daarnaast is het integreren met externe systemen niet altijd evident. Indien het extern systeem zelfs gestuurd kan worden door een orkestratiesysteem, dan is dit extern systeem zich niet bewust van het totaal proces of zijn plaats hierin. Vendors van externe systemen bieden vaak integraties aan met andere bekende systemen, maar dit is veelal duur en maakt het IT-systeem van het bedrijf meer een zwarte doos voor het intern team met hoog risico van vendor lock-in. Een andere uitdaging is menselijk. Het invoeren van een orkestratie systeem kan stuiten op weerstand omdat het werk daardoor niet enkel geroutineerder wordt, maar het veelal problemen blootlegt die men op allerlei niveaus van de onderneming liever zou negeren. Een sterk change management traject is vaak nodig om deze valkuilen te overbruggen. \autocite[pp. 360-365]{Dumas2018} \newline

\section{\IfLanguageName{dutch}{Proces Monitoring en Orkestratie als IT-systeem}{Proces Monitoring en Orkestratie als IT-systeem}}%
\label{sec:proces monitoring en orkestratie als IT-systeem}
Zoals hiervoor aangehaald is de theoretische architectuur heel duidelijk en rechtlijnig. Een executie engine bevraagt de monitoring voor data rond de courante proces instanties. Door deze naast het procesmodel te leggen, kan de engine gefundeerd zowel medewerkers als externe systemen aansturen door middel van respectievelijk een takenlijst en API-calls. Er zijn veel opties om dergelijk systeem te bouwen, maar algemeen bekeken zijn er twee courante strekkingen op de markt. \newline

Enerzijds is er de monoliet aanpak die gevolgd wordt door bijvoorbeeld IBM’s Business Process Manager product of de producten van Bizagi. Hierbij zijn de monitoring tools, de procesmodel tool, de executie engine en de takenlijst allemaal verbonden aan elkaar als deel van een groot softwarepakket dat draait op dezelfde server. De proces analisten bouwen in de tool de afgesproken processen na en de nodige front-end façades worden gebouwd zodat alle actoren rechtstreeks kunnen werken op het systeem en hun aansturing kunnen krijgen. Door middel van API-calls kan het systeem dan de uitgaande requests en binnenkomende responses nakijken om zicht te krijgen op de activiteiten van het proces die buiten zijn eigen systeem vallen. \newline

Dit soort design is typisch voor oudere enterprise pakketten en vindt zijn inspiratie dan ook bij de lange geschiedenis van ERP, CRM, SCM en PLM-pakketten. Hoewel dit voor moderne softwareontwikkelaars oubollig kan lijken, zijn de voordelen niet te onderschatten. Doordat de databasis voor zowel monitoring, model als executie engine dezelfde is, kunnen er sterke optimalisaties gedaan worden op databasis managementsysteem niveau zodat dit geen vertragende factor kan zijn. Verder is het feit dat alles een geïntegreerde monoliet is ook zeer nuttig om een lage latency en snelle dataverwerking te garanderen. Gezien de input van de medewerkers ook rechtstreeks in het systeem gebeurt, kan de engine dit ook veel sneller verwerken. \newline

Deze aanpak heeft echter ook nadelen. Dergelijke systemen kunnen moeilijk horizontaal schalen gezien extra instanties starten om de werklast te balanceren nutteloos is. Men zou al processen moeten verspreiden over verschillende systemen die elk verantwoordelijk zijn voor hun eigen domein om een soortgelijk effect te bekomen. Verticaal schalen is uiteraard een optie, maar die is qua kostprijs niet ideaal. Verder is het uitbreiden van dergelijke systemen niet makkelijk. Als het in-house gebouwd systeem is, dan zal elke uitbreiding bijdragen tot de groeiende complexiteit die gekend is bij monolieten en indien het aangekocht is, dan zijn extensies altijd duur. De laatste valkuil hierbij is dan ook vendor lock-in. Na de setup van dergelijk systeem is een migratie vaak een zeer duur en moeizaam project. \autocite[p. 348]{Dumas2018} \newline

Een andere en modernere aanpak is de gedistribueerde aanpak op basis van hedendaagse microservices architectuur met eventing. In dit model vormen de proces monitoring tool en de proces orkestratie executie engine hun eigen microservice met gelinkte databasis. Elke front-end waar een eindgebruiker in werkt en back-end service die een rol speelt in het proces krijgt dan een library die hen toelaat om events te genereren en om die via een event broker op een queue te plaatsen. De proces monitor tool zal deze events dan kunnen consumeren om zo analoog aan de event logging van een monolithisch systeem een beeld te krijgen van waar de verschillende procesinstanties zich op dit moment bevinden en om deze data bij te houden voor zowel online als offline proces monitoring. De proces orkestratie tool zal dan analoog aan de executie engine zowel de proces monitoring tool als een proces documentatie tool bevragen die als bron van waarheid fungeert voor het hele proces. Op basis van deze input kan deze dan zowel externe services als medewerkers aansturen door via API-calls de nodige requests te sturen naar beiden. \newline

De voordelen van deze aanpak zijn uiteraard gelijkaardig aan de algemene voordelen van microservices architectuur met eventing die zeer standaard is in hedendaagse softwareontwikkeling. Door kleine services met specifieke verantwoordelijkheden te gebruiken verhogen we onze schaalbaarheid via horizontale werklast balanceren door extra instanties te starten die naar dezelfde databasis schrijven. Onderhoudbaarheid is dan ook hoger omdat de services kleiner zijn dan een grote monoliet, waardoor deze door meerdere teams onderhouden kunnen worden met snellere verbeter en release cycli. Door gebruik van eventing is deze architectuur ook veel robuuster. Er kan immers een microservice falen zonder impact te hebben op de rest van de architectuur en met minimaal dataverlies omdat events buiten de microservice bewaard worden in de event queue van de broker. Een specifiek voordeel aan deze aanpak is dat de proces monitor en proces orkestratie tool niet meer hard aan elkaar gekoppeld zijn. Deze ontkoppeling geeft ons meer opties om beide services bloot te stellen als API’s voor mogelijke afnemers voor bijvoorbeeld business intelligence rapportage of data science analyse. We zijn niet meer gebonden aan de façades van het BPMS-systeem en kunnen het systeem veel makkelijker integreren in onze eigen front-ends en back-ends om zo winsten te behalen op vlak van UI/UX-design en performante moderne frameworks. Een laatste voordeel is dat we de optie behouden om naast onze eigen tools ook out-of-the-box tools in deze opstelling op te nemen zonder in te boeten aan flexibiliteit. Als blijkt dat het nuttiger is om bijvoorbeeld de proces monitor of procesmodel tool te kopen, dan kan dit gerust mits deze een flexibele API aanbiedt die voldoet aan onze vereisten qua data. \newline

Qua nadelen verliezen we vooral de snelheid en low-latency van een monoliet op een on-premise server. Ze gaan immers geen deel meer uitmaken van hetzelfde systeem, maar onrechtstreeks met elkaar communiceren via events of rechtstreeks via API-calls. Afhankelijk van hoe dit systeem is opgezet zal elke microservice veelal in eigen containers of clusters draaien on-premise of in de cloud. Dit zal logischerwijs zorgen voor een hogere latency. Wat we winnen aan flexibiliteit en onderhoudbaarheid, verliezen we in complexere architectuur waarbij een probleem end-2-end traceren moeilijker wordt. \autocite[pp. 8-22]{Janiesch2012} \newline

De vraag is dan ook welk van deze twee aanpakken het beste past bij de specifieke use-cases van onze onderzoeksvraag. Beiden zouden werken, maar gelet op de specifieke situatie van het casusbedrijf, lijkt de microservice architectuur een logischere keus. Het casusbedrijf heeft immers veel custom development binnen zijn gebruikte software waar makkelijker op kan worden ingehaakt net omdat de code kan worden aangepast om events te versturen als reactie op specifieke activiteiten die het procesmodel volgen. Voor software die niet in-house werd gemaakt, bestaat oftewel de optie om die te configureren om REST-calls te sturen na specifieke activiteiten of is er een custom integratie naar de eigen systemen die dit kan doen. Een laatste belangrijke consideratie is dat het casusbedrijf intern draait op een microservices architectuur met eventing, waardoor dit systeem volledig zou passen binnen het architecturaal model en zou kunnen inhaken op de bestaande infrastructuur rond eventing en containerisatie in de cloud. Verdere analyse rond requirements en technische vereisten zal echter uitwijzen of dit daadwerkelijk de meest gepaste aanpak \newline

%%=============================================================================
%% Analyse
%%=============================================================================

\chapter{\IfLanguageName{dutch}{Analyse}{Analysis}}%
\label{ch:analyse}
\section{\IfLanguageName{dutch}{Casusbedrijf}{Casusbedrijf}}%
\label{sec:casusbedrijf}
Het casusbedrijf Liantis is een middelgrote onderneming met meer dan 2000 medewerkers. Het vindt historische zijn roots in 1942 als ADMB Sociaal Bureau, waar het werd opgericht om ondernemers te ondersteunen in kader van sociale zaken en personeelsbeleid als sociaal secretariaat. Een latere fusie in 2018 met Zenito, een vzw die diensten aanbood aan beginnende zelfstandigen en Previkmo, een externe dienst voor preventie en bescherming op het werk, creëerde de huidige dienstengroep Liantis.\newline

Deze fusie ging uiteraard gepaard met een grote oefening rond procesbeheer en IT. De drie afzonderlijke IT-departementen werden, samen met hun portfolio aan applicaties, gefuseerd tot een IT-bedrijf specifiek gericht op ontwikkeling, implementatie en onderhoud van het IT-landschap van de volledige dienstengroep. Dit verklaart dan ook de hoge hoeveelheid aan in-house ontwikkeling. Elk afzonderlijk bedrijf had immers voor zijn eigen werking software ontwikkeld en ze waren allen te kleinschalig om grotere out-of-the-box systemen te implementeren. Dit versprokkeld IT- en proceslandschap blijft voor dit bedrijf tot op de dag van vandaag wegen op zijn groei en bemoeilijkt het moderniseren van zijn werking.
\section{\IfLanguageName{dutch}{Functionele Vereisten}{Functionele Vereisten}}%
\label{sec:functionele vereisten}
\subsection{\IfLanguageName{dutch}{Opzet}{Opzet}}%
\label{subsec:opzet func}
Om te ontdekken welke mogelijke toepassingen dit systeem kan hebben, werden workshops georganiseerd om een aantal stakeholders langs business kant te bevragen. De workshops werden als volgt georganiseerd:
\begin{itemize}
  \item De stakeholders kregen een brede uitleg met wat proces monitoring en orkestratie is met voorbeelden vanuit het werkveld. Gelet op hun ervaringen met andere soorten automatisatie pakketten konden ze snel begrijpen hoe dergelijk systeem in theorie werkt.
  \item De stakeholders werden gevraagd om in groep te brainstormen rond mogelijk praktische applicaties van dergelijke systemen binnen het bedrijf. Ze kennen immers de huidige producten het beste als domeinexperten.
  \item De stakeholders moesten post-its met hun theoretische use-cases voor zowel monitoring als orkestratie op het bord zetten met eveneens hoe zij dit in praktische implementaties zouden gerealiseerd willen zien binnen de bestaande producten.
\end{itemize}
De bedoeling van deze workshop was enerzijds om te polsen welke noden dit systeem zou kunnen vervullen, maar ook om te kijken waar het zou kunnen inhaken op de al bestaande systemen. Als bepaalde systemen een duidelijke nood hebben, dan zou het proces die deze systemen hanteert een sterk kandidaat-proces zijn voor de proof-of-concept. Als stakeholders werd gekozen om een business architect, een business owner, een productmanager en een aantal product owners te betrekken bij deze workshops. Hierdoor was er een goede mix van management, architectuur en domeinkennis beschikbaar, maar is er ook een praktische link naar het IT-landschap aanwezig voor input.


\subsection{\IfLanguageName{dutch}{Workshop Proces Monitoring}{Workshop Proces Monitoring}}%
\label{subsec:workshop proces monitoring}
\begin{center}
  \captionsetup{type=figure}
  \includegraphics[width=1.0\linewidth]{PM.png}
  \captionof{figure}[Whiteboard voor de proces monitoring workhop]{Whiteboard voor de proces monitoring workhop}
\end{center}
De workshop proces monitoring onthulde een aantal mogelijke use-cases met bijhorende realisaties binnen het IT-landschap.\newline

De eerste grote use-case is de vraag om op basis van de monitoring zowel de klant als bevoegde manager op de hoogte te houden van het lopend proces. Business zou hierbij willen dat er een timeline is van elk lopend proces, dat het duidelijk is wie momenteel bezig is aan welke stap binnen elk instantie van het proces en of bepaalde mijlstenen van het proces bereikt zijn. Een subset van deze informatie zou nuttig zijn om de klant in real-time te informeren over het lopend proces. Dit zou enerzijds gerealiseerd kunnen worden met een statuspagina binnen het klantenportaal MyLiantis per klant en anderzijds een diepere leidinggevende view binnen de Digitale Cockpit waar de manager een helikopterzicht zou krijgen van de werkzaamheden van het team. \newline

Een tweede grote use-case is de vraag naar betere rapportage en complexe business intelligence. Business zou graag een dashboard willen waar statistische data rond processen beschikbaar is zoals doorlooptijden, maar ook waar key performance indicatoren gemeten worden. Op basis van deze data zou ook de werklast van elke medewerker kunnen gevolgd worden om ervoor te zorgen dat die efficiënt benut wordt, maar ook om overwerk te vermijden. Op basis van de groeiende historische data zou men ook aan forecasting willen doen en zelfs de verwachte aflooptijd willen kunnen berekenen van een lopend proces. Tot op heden is dergelijk dashboard niet voorhanden. \newline

Een derde grote use-case is anomaliedetectie. Op basis van vooraf bepaalde servicelevel overeenkomsten per activiteit moet het mogelijk zijn om bij een potentiële breuk op deze overeenkomsten automatisch de juiste mensen te contacteren en via escalatiepaden remediërende acties in gang te steken zoals reminder-e-mails of herprioretisering van taken.\newline

Het is duidelijk dat er veel mogelijke toepassingen zijn voor de data die proces monitoring zou capteren mits deze op een correcte manier beschikbaar gesteld kan worden via nuttige API’s. Enerzijds zou dit real-time status info voor business en klanten kunnen faciliteren en anderzijds zou dit verregaande rapportage op basis van business intelligence toelaten met eventuele anomaliedetectie. Opvallend is dat het klantenportaal MyLiantis vaak terugkeert in de antwoorden. Dit portaal heeft als centraal proces de onboarding van nieuwe medewerkers bij klanten. Dit doet vermoeden dat het onboarding proces voor nieuwe medewerkers van klanten een interessant kandidaat proces kan zijn voor de proof-of-concept.

\subsection{\IfLanguageName{dutch}{Workshop Proces Orkestratie}{Workshop Proces Orkestratie}}%
\label{subsec:workshop proces orkestratie}
\begin{center}
  \captionsetup{type=figure}
  \includegraphics[width=1.0\linewidth]{PO.png}
  \captionof{figure}[Whiteboard voor de proces orkestratie workhop]{Whiteboard voor de proces orkestratie workhop}
\end{center}

De workshop proces orkestratie onthulde een aantal mogelijke use-cases met bijhorende realisaties binnen het IT-landschap. \newline

De eerste grote use-case is de vraag naar een manier om een slimme en data-gestuurde volgorde in de uitvoering van taken door medewerkers te implementeren met oog op werklast optimalisatie. Het systeem zou op het juiste moment de juiste taak moeten suggereren voor elke medewerker op basis van het aantal openstaande dossiers, de relevante servicelevel overeenkomsten en hun huidige werklast. Een mogelijke realisatie hiervoor zou dan een takenoverzicht per dossier zijn in het klantenportaal MyLiantis en een persoonlijke takenlijst in de Digitale Cockpit van elke medewerker. \newline

Een tweede use-case zou een manier moeten zijn om in bulk te kunnen ingrijpen op dit systeem als manager indien er speciale situaties zich voordien waar het systeem geen rekening mee kan houden op basis van de info vanuit het proces en de monitoring. Een voorbeeld hiervan zijn VIP-klanten of taken die omwille van speciale uitzonderingsevents zoals wettelijke veranderingen opeens een harde deadline krijgen. Dit heeft twee mogelijke implementaties. Enerzijds zou de manager dit imperatief willen kunnen doen door naar een specifiek dossier te gaan en daar de volgende openstaande taak binnen het proces toe te wijzen aan een medewerker. Anderzijds zou een manager specifieke dossiers of zelfs klanten willen kunnen prioriteren zodat hun taken of dossiers voorrang krijgen op die van anderen. Een mogelijke realisatie hiervoor zou een uitbreiding zijn op de voorgenoemde manager-view in de Digitale Cockpit waarbij de manager naast de rapportage vanuit de proces monitoring ook deze aanpassingen zou kunnen doen voor dossiers die binnen zijn portfolio passen. \newline

De nood aan een taakoverzicht voor medewerkers is een heel bekende bij bedrijven en is een van de features waar de systemen van IBM en Bizagi mee uitpakken. Dit zou dan ook een heel basic vereiste moeten zijn. De nood om taken imperatief te kunnen toewijzen aan medewerkers is ook zeker te verantwoorden. Een actieve manager zal de prioriteiten altijd beter kunnen inschatten dan een automatisch systeem. Prioriteiten voor dossiers en klanten aanpassen is echter een veel complexere eis omdat dit rechtstreeks het algoritme van de executie engine zou affecteren. Dit zou alvast iets zijn voor een verdere iteratieve cyclus en niet voor een proof-of-concept of MVP. \newline

Bij deze workshop kwam het klantenportaal MyLiantis nogmaals een aantal keer aan bod. Dit bevestigt dat het onboarding proces een interessant kandidaat proces kan zijn voor ons.

\section{\IfLanguageName{dutch}{Technische Vereisten}{Technische Vereisten}}%
\label{sec:technische vereisten}
\subsection{\IfLanguageName{dutch}{Opzet}{Opzet}}%
\label{subsec:opzet tech}
Nu de functionele vereisten veel duidelijker zijn, komen de technische vereisten en niet-functionele vereisten aan bod. Om deze te capteren werd een brainstormingsessie georganiseerd met een van de systeem architecten van het casusbedrijf die de materie rond proces monitoring en orkestratie goed kent. De eerder gevonden vereisten werden besproken en op basis hiervan werd nagegaan aan welke vereisten het systeem minimaal zou moeten voldoen om te passen binnen het IT-landschap van het bedrijf.
\subsection{\IfLanguageName{dutch}{Brainstorming Sessie met Systeem Architect}{Brainstorming Sessie met Systeem Architect}}%
\label{subsec:brainstorming sessie met systeem architect}
Het casusbedrijf gebruikt in al zijn softwareapplicaties een combinatie van Spring Boot voor de back-end en Angular voor de front-end. Qua programmeertalen zijn de standaardtalen enerzijds Java en anderzijds Type-Script. Op vlak van databasissen is de standaard vooral traditionele SQL, hoewel er geen probleem is om No-SQL opties te gebruiken zoals de documentstore Cosmos DB. Andere No-SQL opties zoals een graaf databasis zijn eerder een uitzondering, maar worden wel ondersteund. \newline

Architecturaal gebruikt het bedrijf vooral een microservices architectuur waarbij een applicatie traditioneel bestaat uit een of meerdere back-ends met bijhorende databasis, een gateway die de toegang tussen back-ends onderling en front-ends medieert en een front-end waar de eindgebruikers in werken. De microservices communiceren met elkaar op basis van REST API-calls indien ze deel uitmaken van dezelfde applicatie of via berichten en eventing als ze dat niet doen. Afnemers van data mogen ook rechtstreeks API-calls uitvoeren via de gateway. De berichten worden geregeld via RabbitMQ, een message broker die het AMQP-standaard ondersteund, terwijl events geregeld worden via Azure Event Hubs.\newline

Infrastructureel zit het casusbedrijf nog volop in zijn transitie naar de cloud. Een deel van de software draait on-premise op een datacenter met Apache Tomcat servers terwijl merendeel van de software draait op Microsoft Azure Cloud in Docker containers. Voor de on-premise software wordt de versionering gedaan via Bitbucket met Jenkins als ci/cd pipeline terwijl de toepassingen in de cloud op Github staan met Github actions als ci/cd pipeline. Voor een zeer klein deel legacy applicaties wordt de versionering en configuratie nog gedaan via Subversion, hoewel deze volop uitgefaseerd worden. \newline

Dit overzicht geeft ons al een duidelijk beeld van waar de software aan gaat moeten voldoen. Als databasis is vrije keuze tussen SQL of een documentstore per back-end. De back-end moet in Java gecodeerd zijn met Spring Boot als framework. Indien er een front-end of meerdere back-ends aanwezig zijn, is er nood aan een gateway om de connectie de mediëren. Indien er een front-end is, dan moeten deze in Type-Script geschreven worden met Angular als framework. Inkomende communicatie vanuit andere systemen gaat via berichten. Afnemers hebben een duidelijke REST-API nodig. \newline

Gelet op de vereisten van business is voor de proof-of-concept een front-end niet aan de orde. Het is immers een vereiste dat de informatie uit beide systemen ingebouwd wordt in al bestaande front-ends. Rapportage gebeurt ook via business intelligence dashboards. Het is dus vooral belangrijk dat de back-end systemen duidelijk API’s hebben voor afnemende systemen en business intelligence. Dit betekent ook dat een gateway niet aan de orde zal zijn. Hoewel er argumenten zijn om te werken met een documentstore, primeert de nood aan consistente en juiste data boven alles. SQL is ook de regel binnen het landschap waardoor No-SQL een nutteloze complexiteit lijkt.  De technische scope is nu dus veel duidelijker.
\section{\IfLanguageName{dutch}{Niet-functionele Vereisten}{Niet-functionele Vereisten}}%
\label{sec:niet-functionele vereisten}
\subsection{\IfLanguageName{dutch}{Overlopen NFR-checklist met Systeem Architect}{Overlopen NFR-checklist met Systeem Architect}}%
\label{subsec:overlopen nfr-checklist met systeem architect}
Het casusbedrijf meet elke potentiële applicatie aan een aantal niet-functionele vereisten. Hierbij moet er altijd het minimum bereikt worden van ISO-norm 25010 als standaard voordat deze toegelaten kan worden als applicatie binnen het IT-landschap. Afhankelijk van de aard van het product zullen de vereisten strenger zijn. Standaard moet de trekker van een project een NFR-checklist invullen voor om het even welk softwareproject mag opgestart worden. Dit moet dan door de architecten gevalideerd worden zodat er minstens aan de minimale vereisten wordt voldaan. Deze checklist werd tijdens de brainstorming sessie ook ingevuld. Voor een proof-of-concept zijn de eisen veel lager gezien het de bedoeling is dat dit tijdens het eigenlijk project wordt behaald. Het theoretisch eindproduct zou echter moeten voldoen aan de volgende niet-functionele vereisten. \newline

Availability is als vereiste verbonden aan hoe kritisch het business proces is dat de software ondersteunt en wat de schade voor het bedrijf is indien de werking van de software verstoord wordt. Uit voorgaande workshops zag business de software enerzijds als informatief voor klant en de manager, anderzijds als sturend voor de medewerkers. Het neergaan van de software zou ertoe leiden dat een aantal processen zwaar vertraagd tot onuitvoerbaar zouden zijn. Hierdoor zou de software bestempeld worden als bedrijf kritisch. Qua availability moet deze software dus tijdens de kantooruren zo goed als 99,999\% van de tijd beschikbaar zijn. Het bedrijf volgt hier het “Five Nines”-principe. He is even streng qua foutbestendigheid. De software moet zichzelf kunnen remediëren op basis van zijn interne logica en data op basis van back-ups. \newline

Performantie is ook belangrijk gezien vertragingen rechtstreeks geld kosten. De algemene verwachting binnen het bedrijf voor synchrone processen is een responsetijd van 100 milliseconden en voor asynchrone processen is dit 30 seconden. Op vlak van presentatie is mag de laadtijd van een scherm niet boven 1 seconde zijn en voor acties binnen een scherm is de verwachting een responstijd van maximaal 500 milliseconden. Dit is algemeen zo voor alle applicaties binnen het bedrijf.\newline

In kader van configureerbaarheid en beheerbaarheid moet het product een externe configuratie kunnen aanvaarden. Een gebruiker hoort de data voor proces monitoring niet van buitenaf te kunnen beheren gezien het systeem zelf zijn data genereert. Ingrijpen op taken binnen de orkestratie zou dan weer wel moeten via een duidelijke UI of API die mogelijke acties van eindgebruikers limiteert tot het essentiële. De data zelf moet voor een gewone lezer niet zomaar uit te lezen zijn en moet, waar mogelijk, werken met generische referenties zodat enkel een ander systeem dit correct kan interpreteren en tonen aan de gebruiker. Dit houdt ook in dat het systeem input moet verwerken van om het welk proces of systeem binnen het bedrijf zonder te weten wat de data zelf representeert. De uiteindelijk vertaling is de verantwoordelijkheid van de afnemer. \newline

Op vlak van observability is er minimaal een technische monitor rond de infrastructuur waar de software op draait voor status, CPU-gebruik, geheugen-gebruik, disk I/O, disk ruimte en netwerk I/O aanwezig. Dit zit al ingebouwd op de cloud containers en moet dus in meeste gevallen niet actief gebouwd worden. Er moet ook duidelijke interne logging zijn die extern aggregeerbaar is via tooling zoals Dynatrace of Grafana. \newline

Security is een belangrijk onderwerp. Het casusbedrijf heeft daarom een intern authenticatie module voor elke login en endpoint. Er moet ook gelaagde autorisatie met rollen ingevoerd zijn in de software. Als de mogelijkheid bestaat om via menselijke interactie de data aan te passen, dan moet er een historische audit log bijgehouden worden waarin staat wie welke data wanneer aanpaste met de oude en nieuwe waarde in de log. 
%%=============================================================================
%% Proof-of-concept
%%=============================================================================

\chapter{\IfLanguageName{dutch}{Proof-of-concept}{Proof of concept}}%
\label{ch:proof-of-concept}
\section{\IfLanguageName{dutch}{Design}{Design}}%
\label{sec:design}
\subsection{\IfLanguageName{dutch}{Architecturaal Model}{Architecturaal Model}}%
\label{subsec:architecturaal model}
In de analyse fase werd duidelijk aan welke technische en architecturale vereisten de software zou moeten voldoen. Voor het architecturaal beeld worden twee diagrammen voorzien met uitleg. Dit omdat het theoretisch systeem aan andere eisen moet voldoen dan de proof-of-concept.
\subsubsection{\IfLanguageName{dutch}{Theoretisch Systeem}{Theoretisch Systeem}}%
\label{subsubsec:architect theoretisch systeem}
\begin{center}
  \captionsetup{type=figure}
  \includegraphics[width=1.0\linewidth]{theoretisch architectuur.jpg}
  \captionof{figure}[Architecturaal model van het theoretisch systeem]{Architecturaal model van het theoretisch systeem.}
\end{center}
Het theoretisch systeem volgt de microservice architectuur van het bedrijf met een duidelijke afscheiding tussen de services met een proces monitoring service en een orkestratie service als executie engine. Beide systemen krijgen hun input via eventing. Zodra de processen bepaalde mijlpalen bereiken zoals gedefinieerd in de BPMN van het proces, produceren de bijhorende systemen een event die op de queue leeft. Deze worden dan geconsumeerd door beide systemen om data rond het proces en om taken voor de medewerkers te genereren. \newline

Binnen het bedrijf wordt de proces documentatie tool ARIS gebruikt om processen te modelleren en te documenteren. Deze heeft een REST-API die het theoretisch systeem kan bevragen. Het proces monitoringsysteem bevraagt ARIS voor info over de processen om context op te bouwen indien nodig. De orkestratie gebruikt dan de monitoring voor context om zijn beslissingen te maken. Beide systemen schrijven naar dezelfde databasis met verschillende schema’s. Deze databasis wordt dan geëxtraheerd naar een datawarehouse voor datamining en business intelligence. Een gateway dient als enige punt van toegang voor externe systemen voor mediatie en extra beveiliging. Op basis van de vragen die binnenkomen op de API van de gateway bevraagt deze de correcte achterliggende systemen, ook als er meerdere instanties van de systemen draaien door horizontale opschaling. \newline

Via deze aanpak wordt een maximale ontkoppeling tussen de systemen gegarandeerd, bestaat de optie om op te schalen bij zwaardere lading op het systeem, is asynchrone dataverwerking door met een event queue te werken bereikt en bestaat de mogelijkheid voor zowel offline als online data analytische onderzoek door tegelijk te werken met een transactioneel databasis dat een historische datadimensie verkrijgt door stelselmatig te extraheren naar een datawarehouse. \newline

\subsubsection{\IfLanguageName{dutch}{Proof-of-Concept}{Proof-of-Concept}}%
\label{subsubsec:architect Proof-of-concept}
\begin{center}
  \captionsetup{type=figure}
  \includegraphics[width=1.0\linewidth]{poc-architectuur.jpg}
  \captionof{figure}[Architecturaal model van het proof-of-concept systeem]{Architecturaal model van het proof-of-concept systeem.}
\end{center}
Voor de proof-of-concept zal de architectuur simpeler zijn daar dit systeem enkel maar moet bewijzen dat het concept werkt. Een proof-of-concept hoort niet in te haken op de event hub, dus deze zal synchroon zijn data krijgen via zijn REST-API in plaats van asynchroon events te verwerken. Dit geeft als voordeel dat voor de tests en demo de input gesimuleerd kan worden via Postman. Monitoring en orkestratie gaan maar één specifiek proces verwerken voor de proof-of-concept waardoor een integratie met de documentatie tool van het bedrijf overbodig is. Er gaan ook geen bestaande afnemers gekoppeld worden om de werking het systeem te valideren, maar dit wordt manueel gedaan door het systeem via zijn API te bevragen. Hierdoor zijn een gateway en datawarehouse ook nodeloze complexiteit.\newline

Door in de proof-of-concept de focus te leggen op de monitor en de executie engine als de centrale componenten van het BPMS-systeem kan snel en efficiënt bewezen worden dat het systeem doet wat verwacht wordt zonder impact op de bestaande infrastructuur en met zo weinig mogelijk extra kosten voor het bedrijf zelf.

\subsection{\IfLanguageName{dutch}{Domein Model}{Domein Model}}%
\label{subsec:domein model}
Binnen het domein staan twee objecten centraal. Een proces monitoringsysteem draait rond het genereren van executie logs die gebruikt worden om de lopende processen te volgen terwijl een orkestratie systeem taken genereert voor medewerkers op het juiste moment en dit beiden op basis van de input van de externe systemen. 

\subsubsection{\IfLanguageName{dutch}{Theoretisch Systeem}{Theoretisch Systeem}}%
\label{subsubsec:domein theoretisch systeem}
\begin{center}
  \captionsetup{type=figure}
  \includegraphics[width=1.0\linewidth]{theoretisch domein.jpg}
  \captionof{figure}[Domein model van het theoretisch systeem]{Domein model van het theoretisch systeem.}
\end{center}
Beide objecten worden in dit domein model opgemaakt, maar staan los van elkaar. Dit is logisch gezien monitoring en orkestratie wel samen werken en elkaar voeden, maar nooit deel uitmaken van elkaar. Bij de opmaak van het domein werden een aantal vereisten opgelegd door de systeem architect gevolgd. \newline

Enerzijds moet het systeem proces-agnostisch zijn. Dit wil zeggen dat de proces monitoring de input van om het even welk systeem en proces binnen het bedrijf moet kunnen aanvaarden en monitoring genereren. De executie logs horen ook zoveel mogelijk de BPMN-standaard te volgen om de interoperabiliteit met bestaande de documentatietool te vergemakkelijken. Een executie log van een proces kan daarom zowel de log van een proces als een sub proces zijn met referentie naar het bovenliggend proces. Een activiteit binnen de log kan een handeling binnen de BPMN-standaard als zijnde een atomair stuk werk voorstellen. Eveneens kan het een beslissing node waarop het verloop kan uitsplitsen in verschillende activiteiten en weer samenkomt voorstellen. Verder kan het ook een sub proces als deel van het bovenliggend proces voorstellen. Door te werken met de types STEP (handeling), SPLIT (uitsplitsing), MERGE (samenkomst) en SUBPROCESS (sub proces) voor elke activiteit worden deze uit elkaar gehouden. De collectie aan activiteiten die sequentieel elkaar opvolgen vormen dan het verloop van dit proces. Hierbij is het architecturaal niet de taak van het systeem om context te geven aan de ruwe data, maar wel aan de afnemer van de data. Die zal dan op basis van de data uit het monitoringsysteem en de data uit de documentatie tool een reconstructie van het procesverloop kunnen maken. \newline

Anderzijds horen zowel de taken als executie logs enkel maar referenties naar data in andere systemen te bevatten. Zodoende kan enkel maar de afnemer, die weet waar hij de juiste data op basis van de referenties moet opvragen, de data reconstrueren. Zo worden immers datalekken binnen het systeem vermeden. Dit zorgt ervoor dat zowel de executie logs als de taken vooral bestaan uit referentie id’s, tijdsvermeldingen en voorgedefinieerde statussen. Het domein bevat de volgende velden met bijhorende uitleg:
\begin{itemize}
  \item processReference: referentie naar het proces in de documentatietool.
  \item processInstanceReference: unieke referentie naar een specifieke run van het proces.
  \item parentProcessInstanceReference: Refereert naar het bovenliggend proces waarvan dit proces een stap is. Dit is enkel ingevuld als het proces een sub proces is.
  \item processStatus: de huidige status van het volledig proces zijnde START, SUCCES of FAIL.
  \item processTimestamp: de tijdsvermelding van de huidige status van het proces.
  \item initiatorReference: referentie naar het object waarvoor dit proces is opgestart. Dit is bijvoorbeeld een klant of een medewerker.
  \item beneficiaryReference: referentie naar object dat onderwerp is van het proces. Dit is bijvoorbeeld een medewerker van een klant of een factuur bij een klant.
  \item providerReference: referentie naar het systeem binnen het bedrijf die bezitter is van het proces. Het proces start en eindigt normaal in dit systeem.
  \item activityReference: referentie naar een activiteit of beslissing node of subproces in de documentatietool indien relevant.
  \item activityInstanceReference: unieke referentie naar deze specifieke run van de activiteit binnen het proces.
  \item subprocessReference: referentie naar een subproces in de documentatietool indien relevant.
  \item subprocessInstanceReference: unieke referentie naar deze specifieke run van het sub proces indien relevant.
  \item activityStatus: de huidige status van de activiteit zijnde START, SUCCES of FAIL.
  \item activityTimestamp: de tijdsvermelding van de huidige status van de activiteit.
  \item executorReference: de huidige uitvoerder van de activiteit, indien relevant. Dit kan een medewerker zijn binnen het bedrijf of een systeem.
  \item type: het soort activiteit zijnde STEP, SPLIT, MERGE of SUBPROCESS.
  \item parameters: vrij veld voor metadata.
  \item taskReference: referentie naar de taak in de documentatietool.
  \item taskInstanceReference: unieke referentie naar deze specifieke run van de taak
  \item startTimestamp: de tijdsvermelding van de start van de taak.
  \item stopTimestamp: de tijdsvermelding van de stop van de taak.
  \item taskStatus: de huidige status van de taak zijnde ONGOING, WAITING, SUCCES, FAIL.
  \item currentTasks: het aantal taken dat de betreffende executor momenteel uitvoert.
  \item maxAllowedTasks: het aantal taken dat een executor mag hebben om overwerk te voorkomen.
  \item allowedTaskReferences: een exhaustieve lijst van type taken dat deze executor mag uitvoeren op basis van zijn kennis, vaardigheden of team.
\end{itemize}
Omdat de executie logs proces-agnostisch moeten zijn, is er veel metadata nodig om te kunnen opzoeken waar in de documentatietool het proces, de activiteit of de taak hoort, van waar de data komt, wie momenteel bezig is aan een taak en voor welke klant het proces draait.  Dit is echter van vitaal belang om enerzijds de juiste data op te kunnen vragen voor de afnemers en anderzijds voor de correcte reconstructie van het verloop bij analyse en business intelligence.
\subsubsection{\IfLanguageName{dutch}{Proof-of-Concept Systeem}{Proof-of-concept Systeem}}%
\label{subsubsec:domein proof-of-concept systeem}
\begin{center}
  \captionsetup{type=figure}
  \includegraphics[width=1.0\linewidth]{poc domein.jpg}
  \captionof{figure}[Domein model van het proof-of-concept systeem]{Domein model van het proof-of-concept systeem.}
\end{center}
Gezien de proof-of-concept veel kleiner in scope is, kan het domein model ook drastisch verkleind worden. De referenties naar de documentatietool zijn verwijderd omdat de proof-of-concept niet moet kunnen integreren hiermee. De referenties naar externe systemen zijn eveneens verwijderd gezien de input gesimuleerd zal worden via Postman. Om verdere nutteloze complexiteit te mijden, kan in de proof-of-concept elke executor alle taken van het simulatie proces uitvoeren en wordt er enkel maar gekeken naar het balanceren van de werklast. Hierdoor bereikt het domein zijn minimale essentiële vorm waarbij voor een specifiek instantie van proces een executie log kan gegenereerd worden die het verloop van het proces kan representeren op een manier die strookt met de BPMN-standaard en eveneens taken kan genereren. 

\subsection{\IfLanguageName{dutch}{Entiteit Relationeel Diagram}{Entiteit relationeel diagram}}%
\label{subsec:entiteit relationeel diagram}
Het domein model wordt verder gebruikt als basis voor het entiteit relationeel diagram. Het diagram is gemaakt voor een SQL-databasis omdat het casusbedrijf aangeeft dat dit de standaard is. Voor zowel het theoretisch systeem als de proof-of-concept lijkt dit dus een correcte beslissing. Het domein model is in ieder geval compatibel met zowel SQL als een No-SQL documentstore. \newline

Gezien de aard van het systeem zou naast een SQL-databasis een documentstore ook zeker op zijn plaats zijn. Het systeem bevat immers vooral referenties naar andere systeem. Intern is er enkel een one-to-many relatie tussen de executie log van het proces en de uitgevoerde activiteiten en een one-to-one tussen taak en uitvoerder. Dit is hierdoor makkelijk te modelleren als een document waarbij de executie log een array aan activiteiten bevat. Volgens het CAP theoretisch model van Brewer is een documentstore een betere keuze voor de proces monitoring gezien de ingestie van brede data aan hoog volume hier primair is. Dit betekent dat er meer waarde is in het schalen van de databasis en in algemene bereikbaarheid van data dan in consistentie. Voor de orkestratie zou SQL echter altijd de juiste keuze zijn gezien consistente taken hebben voor de medewerkers belangrijker is dan schalen.  Bij de opzet van het theoretische systeem zou het zeker een waardevolle oefening zijn om de voordelen en nadelen van een volledig SQL-databasis versus een hybride oplossing te onderzoeken.

\subsubsection{\IfLanguageName{dutch}{Theoretisch Systeem}{Theoretisch Systeem}}%
\label{subsubsec:erd theoretisch systeem}
\begin{center}
  \captionsetup{type=figure}
  \includegraphics[width=1.0\linewidth]{theoretische ERD.jpg}
  \captionof{figure}[ERD van het theoretisch systeem]{Entiteit Relationeel Diagram van het theoretisch systeem.}
\end{center}
Dit model is opgebouwd uit vier complexe tabellen die vooral referentie data, statussen en tijdsvermeldingen bevatten. Dit volgt de redenering dat het systeem zijn eigen data niet moet kunnen interpreteren en dit moet overlaten aan andere systemen. De executie log heeft een one-to-many relatie met zijn activiteiten via een tussentabel terwijl taken een one-to-one relatie heeft met zijn uitvoerder. Dit is nodig zodat het systeem taken aan de correcte medewerkers qua werklast en toegelaten taken kan uitdelen. Het systeem moet dus een simpele subset reflectie bevatten van de master data omtrent wie welke taken mag uitvoeren en bijhouden hoeveel taken deze mensen momenteel uitvoeren versus hun maximaal toegelaten lading.

\subsubsection{\IfLanguageName{dutch}{Proof-of-Concept Systeem}{Proof-of-concept Systeem}}%
\label{subsubsec:erd proof-of-concept systeem}
\begin{center}
  \captionsetup{type=figure}
  \includegraphics[width=1.0\linewidth]{poc erd.jpg}
  \captionof{figure}[ERD van het proof-of-concept systeem]{Entiteit relationeel diagram van het proof-of-concept systeem.}
\end{center}
Het diagram voor de proof-of-concept volgt dezelfde structuur als die van het theoretisch systeem, maar is uiteraard versimpeld door de kleinere scope van het systeem. Dezelfde informatie die in het domein model wordt weggelaten is hier ook afwezig.

\subsection{\IfLanguageName{dutch}{REST-API}{REST-API}}%
\label{subsec:rest-api}
De technische vereisten van het casusbedrijf vereisen een duidelijke en veilige API voor afname van data. Dit omdat het in eerste instantie niet gewenst is dat een eindgebruiker of extern systeem rechtstreeks kan ingrijpen op de generatie van executie logs of taken. In een latere iteratie van het systeem zal dit echter wel nodig zijn omdat business aangaf dat de optie er moet zijn om expliciet in te grijpen op taken door bepaalde dossiers voorrang te geven of om taken rechtstreeks toe te wijzen aan specifieke medewerkers. 

\subsubsection{\IfLanguageName{dutch}{Theoretisch Systeem}{Theoretisch Systeem}}%
\label{subsubsec:api theoretisch systeem}

Het theoretisch systeem heeft geen nood aan een API om data te ontvangen. De data zal immers komen vanuit een event of message op een hub of broker waar de data leeft tot het systeem de data consumeert. Zowel de data rond het afwerken van stappen in het proces als die van taken zal zo het systeem bereiken. Het architecturaal principe dat systemen zoveel mogelijk met elkaar moeten communiceren via events en messages als ze niet in elkaars domein zitten blijft behouden. Proces monitoring en orkestratie kan immers door zijn doel enkel deel uitmaken van zijn eigen domein. \newline

Voor afname zal er vooral nood zijn aan lees eindpunten met specifieke gelaagdheid rond de data objecten, maar ook aan export eindpunten voor extractie richting de datawarehouse.
 
\begin{center}
  \captionsetup{type=figure}
  \includegraphics[width=1.0\linewidth]{theoretisch monitoring api.png}
  \includegraphics[width=1.0\linewidth]{theoretisch task api.png}
  \captionof{figure}[API-diagram van het theoretisch systeem]{API-diagram van het theoretisch systeem.}
\end{center}

\subsubsection{\IfLanguageName{dutch}{Proof-of-Concept Systeem}{Proof-of-Concept Systeem}}%
\label{subsubsec:api proof-of-concept systeem}
De proof of concept volgt een heel gelijkaardig stramien aan het theoretisch systeem voor afname. Door zijn gelimiteerde scope zijn de eindpunten die te maken hebben met filtering op specifieke processen en business objecten weggelaten gezien er maar één proces zonder gelaagdheid aan business items gevolgd zal worden. De eindpunten voor extractie richting de datawarehouse zijn bijgevolg ook overbodig. Omdat de proof-of-concept zijn input zal krijgen via simulatie zal deze wel eindpunten hebben waar simulatieberichten aankomen met dezelfde inhoud als de messages die het theoretisch systeem zal gebruiken. Er is ook een eindpunt voorzien die ons toelaat om bepaalde taken te markeren als voltooid of wachtende opdat we het taakoverzicht correct kunnen testen.
 
 
\begin{center}
  \captionsetup{type=figure}
  \includegraphics[width=1.0\linewidth]{poc monitoring api.png}
  \includegraphics[width=1.0\linewidth]{poc task api.png}
  \captionof{figure}[API-diagram van het proof-of-concept systeem]{API-diagram van het proof-of-concept systeem.}
\end{center}
\section{\IfLanguageName{dutch}{Implementatie}{Implementatie}}%
\label{sec:implementatie}
Spring Boot is gemaakt voor snelle prototyping van microservices en REST-API’s waardoor het bouwproces zeer pijnloos was. De mogelijkheid om een in-memory databasis te gebruiken biedt dan ook extra voordelen voor het bedrijf gezien er geen nodeloze infrastructuur opgezet moet worden voor een aparte databasis die nadien toch verwijderd wordt.\newline

Het monitoring domein was door de gemaakte analyse betrekkelijk eenvoudig te implementeren. Het gros van de logica dient om de ruwe monitoring data te valideren en correct te interpreteren om de executielog per procesinstantie stelselmatig op te bouwen. Het moet daarom vooral in staat zijn om snel en consistent data te verwerken met oog op foutbestendigheid en correcte logging. Door de context van de data af te schermen van het systeem en de interpretatie hiervan over te laten aan de afnemer kan het systeem zijn focus leggen op zijn singuliere verantwoordelijkheid als verwerker van grote hoeveelheden aan data. \newline

Het taak domein definiëren en correct taken laten generen ging door de analyse ook zeer vlot. De grootste uitdaging was het correct inregelen van de executie engine. Bij out-of-box systemen waarbij de modelering tool rechtstreeks vasthangt aan het monitoringsysteem en de executie engine wordt de mapping tussen taak en activiteit gedaan tijdens het modelleren van het proces. Bij deze microservice aanpak waarbij de modelering tool apart blijft, moet deze mapping expliciet gedaan worden in de configuratie. Dit bleek vrij uitdagend gezien er naast een BPMN-diagram van het te simuleren proces ook een mapping moest zijn van de berichten die per milestone gingen gestuurd worden en welke taken deze potentieel kunnen genereren. Dit was nodig voor de uiteindelijke mapping van monitoring data richting taken om zo de executie engine in te regelen. \newline

Het systeem moest ook minimale kennis hebben over alle medewerkers en het aantal die ze mogen uitvoeren. Het systeem kan zo bijhouden hoeveel actieve taken een medewerker heeft om zo de werklast aan taken evenredig te verdelen. Dit werd opgelost door dummy data te injecteren in de in-memory databasis en logica te schrijven voor deze werklast balancerende werking. Deze twee elementen zorgden voor een verrassende extra laag aan complexiteit.

 \begin{center}
  \captionsetup{type=figure}
  \includegraphics[width=1.0\linewidth]{product uml crop.JPG}
  \captionof{figure}[UML klasse-diagram van het proof-of-concept systeem]{UML klasse-diagram van het proof-of-concept systeem.}
\end{center}

\section{\IfLanguageName{dutch}{Validatie}{Validatie}}%
\label{sec:validatie}
Om te valideren of het systeem werkt, wordt het verloop van een bestaand proces binnen het casusbedrijf gesimuleerd. Hierbij komen via de REST-API berichten aan bij de proof-of-concept systeem die gelijkaardig zijn aan de monitoring events die het theoretisch systeem zal verwerken. Op basis van deze data moet het systeem in staat om de data van het lopend proces correct te bundelen, bloot te stellen aan afnemers en niet enkel taken te generen, maar deze ook toe te wijzen aan onze dummy medewerkers.

\begin{center}
  \captionsetup{type=figure}
  \includegraphics[width=1.0\linewidth]{test proces.jpg}
  \captionof{figure}[BPMN-diagram van het gesimuleerd intern proces]{BPMN-diagram van het gesimuleerd intern proces.}
\end{center}

Om deze validatie correct te laten verlopen werd bij elke node binnen het diagram een mapping gemaakt waarbij wordt gedefinieerd welk monitoring data dat het systeem zal ontvangen binnen de simulatie en welke taak, indien relevant, dient genereerd te worden vanuit het systeem.  Op basis hiervan werd de simulatie data opgebouwd en werd nagegaan de resulterende executielog en de gegenereerde taken voldoen aan de vereisten van het bedrijf om te valideren of de proof-of-concept succesvol is. Om de validatie te vergemakkelijken werd ook een rudimentair front-end gebouwd voor visualisaties.

\begin{center}
  \captionsetup{type=figure}
  \includegraphics[width=1.0\linewidth]{node bpmn.png}
  \captionof{figure}[BPMN-node met mapping]{Een node uit het diagram met bijhorende simulatie data en verwachtte taak.}
\end{center}
 
\begin{center}
  \captionsetup{type=figure}
  \includegraphics[width=1.0\linewidth]{bpmn met mapping.png}
  \captionof{figure}[BPMN-diagram met mapping]{Het volledig diagram met bijhorende simulatie data en verwachtte taken per node.}
\end{center}

\begin{center}
  \captionsetup{type=figure}
  \includegraphics[width=1.0\linewidth]{postman data.png}
  \captionof{figure}[Simulatie-data Postman]{Simulatieflow voor data vanuit externe systemen in Postman.}
\end{center}
Bij aankomst van de initiële monitoring data maakt het systeem een Proces Instantie Executie Log aan die de algemene data van het proces bevat zoals zijn unieke referentie, wie het proces heeft gestart en de huidige status. De Log bevat ook een collectie aan activiteiten die het verloop van het proces sequentieel voorstelt. Zodra er nieuwe data binnenkomt, zal het systeem deze correct interpreteren, mappen in de juiste vorm en toevoegen aan de collectie. De activiteiten volgen de BPMN-standaard door gebruik te maken van de types STEP, SPLIT, MERGE en SUBPROCES. \newline

Als een activiteit ook een sub proces start, dan worden die bi-directioneel aan elkaar gekoppeld. 
\begin{center}
  \captionsetup{type=figure}
  \includegraphics[width=0.75\linewidth]{postman subproces.png}
  \captionof{figure}[Subproces registratie data]{Start, koppeling aan ouder, afloop en registratie van resultaat bij sub en ouder proces.}
\end{center}
Deze data kan dan opgevraagd worden door de afnemers via een API voor implementatie in hun applicaties. In onderstaande data is zichtbaar hoe de uitvoerder een in dienst begon te registreren om 15:55 en klaar was om 16:05. Deze data kan gebruikt worden om de status van het proces terug te koppelen aan de klant, om de gemiddelde doorlooptijd van deze activiteit te berekenen of om anomalie of bottleneck detectie uit te voeren op basis van servicelevel akkoorden. 
\begin{center}
  \captionsetup{type=figure}
  \includegraphics[width=1.0\linewidth]{JSON output api.png}
  \captionof{figure}[JSON-output API Monitoring]{JSON-output van het test-proces indien opgevraagd via de API.}
\end{center}
 
Het systeem is tevens ook in staat om een gepaste taak te genereren op het juiste moment en deze te koppelen aan een gepaste medewerker. Het weet hoeveel taken iedereen heeft en kan daardoor aan load balanceren doen. Dit is ook makkelijk op te vragen via de API.\newline

In onderstaande data zien we hoe als reactie op de activiteiten “REGISTRATIE SOLLICITANT”, “CONTRACT MAKEN” en “CONTRACT BEZORGEN” er drie verschillende taken met eigen traceerbare referenties gemaakt zijn en verbonden werden aan drie unieke medewerkers.
\begin{center}
  \captionsetup{type=figure}
  \includegraphics[width=1.0\linewidth]{nodes met taken.png}
  \captionof{figure}[Nodes met taken]{De relevante nodes uit het diagram met bijhorende simulatie data en verwachtte taken.}
\end{center}
\begin{center}
  \captionsetup{type=figure}
  \includegraphics[width=1.0\linewidth]{taken.png}
  \captionof{figure}[SON-output API Orkestratie]{JSON-output van de drie verwachte gegenereerde taken gekoppeld aan drie verschillende medewerkers.}
\end{center}
\section{\IfLanguageName{dutch}{Visualisatie van Mogelijke Realisaties voor het Casusbedrijf}{Visualisatie van mogelijke realisaties voor het casusbedrijf}}%
\label{sec:visualisatie}
Op basis van de data die uit de API van de applicatie komt, werden rudimentaire front-end visualisaties gebouwd om te valideren of de use-cases die business in de analyse aanhaalde nu technisch mogelijk zijn. \newline

Op basis van de data in de executie logs binnen het systeem is het mogelijk om voor elke lopende instantie van het proces een tijdslijn op te bouwen. Zodoende kan de klant geïnformeerd worden over het verloop en de huidige status van het proces. \newline

Gezien elke activiteit ook oproepbaar is op basis van zijn activiteit referentie en instantie referentie is het mogelijk om alle instantie logs van een specifieke type activiteit op te roepen, de doorlooptijd per instantie te berekenen en statistisch onderzoek te plegen naar doorlooptijden of anomaliedetectie te doen.
\begin{center}
  \captionsetup{type=figure}
  \includegraphics[width=0.89\linewidth]{timeline proces.png}
  \captionof{figure}[Executie log timeline]{Timeline van de lopende instantie van proces abefce68-4ba9-4de2-a3e7-102275f6514b met diens activiteiten in chronologische volgorde.}
\end{center}

Eerder werd de eis gesteld dat het systeem correct moet omgaan met het splitsen van de timeline in parallelle of alternatieve flows. Uit de visualisatie is duidelijk dat hier een SPLIT gebeurd in verschillende parallelle activiteiten die dan later weer samenkomt. Door de data zo te segmenteren is het mogelijk om statistisch onderzoek te doen naar bottlenecks in gesplitste flows.

\begin{center}
  \captionsetup{type=figure}
  \includegraphics[width=0.75\linewidth]{upstream split.png}
  \captionof{figure}[BPMN-diagram upstream split]{Upstream split volgens het BPMN-diagram.}
\end{center}

 
\begin{center}
  \captionsetup{type=figure}
  \includegraphics[width=1.0\linewidth]{upstream visual 1.png}
  \includegraphics[width=1.0\linewidth]{upstream visual 2.png}
  \captionof{figure}[Upstream split visualisatie]{Start en stop van de split als 15de en 24ste stap in de timeline die doorlooptijd van de split berekenbaar maakt.}
\end{center}

Tijdens de split worden er ook sub processen gestart als deel van de activiteiten. Het systeem kan dit correct interpreteren, de processen starten en deze koppelen aan het bovenliggend proces om makkelijke navigatie van de data mogelijk te maken. Volgens bovenstaande BPMN-diagram is de activiteit “MAAK TELLERS AAN” een sub proces. Dit is correct gereflecteerd in de tijdslijn.

\begin{center}
  \captionsetup{type=figure}
  \includegraphics[width=1.0\linewidth]{subproces als activivty.png}
  \includegraphics[width=1.0\linewidth]{subproces als log.png}
  \captionof{figure}[Subproces als traceerbare activiteit en executie log]{Het sub proces “TELLERS AANMAKEN” bestaat als zowel traceerbare activiteit instantie b0cb6e67-b639-4114-b8b7-97450e5a2ede als sub proces instantie 5999d099-ae21-4ff6-b4db-1199c2b72a17. Het sub proces kent zijn bovenliggende proces instantie 6ea084d3-572b-45d9-b01f-28c38076669a.}
\end{center}
 
Als volgens het proces er een taak dient genereerd te worden tijdens de activiteit, dan zal het systeem dit ook doen. Het zal deze taak ook slim uitdelen. Indien de monitoring aangeeft dat er iemand explicit bezig is aan een activiteit, dan zal de executie engine de taak genereren voor die specifieke medewerker zolang zijn werklast dit aankan. In elk ander geval zal het slim kijken naar de werklast van alle medewerkers en deze balanceren zodat iedereen evenveel taken heeft. \newline

De taken die te doen zijn kunnen opgehaald worden per medewerker om zo een takenoverzicht te maken voor de medewerker.

\begin{center}
  \captionsetup{type=figure}
  \includegraphics[width=1.0\linewidth]{bpmn taak ddt.png}
  \captionof{figure}[BPMN-diagram "contract maken" taak]{Het BPMN-diagram geeft aan dat activiteit “CONTRACT MAKEN” de taak “Maak contract aan” dient te genereren.}
\end{center}

\begin{center}
  \captionsetup{type=figure}
  \includegraphics[width=1.0\linewidth]{json ddt.png}
  \captionof{figure}[JSON-output taak "contract maken" met toewijzing]{De monitoring gaf aan dat medewerker “Dimitri De Tremmerie” hieraan werkt.}
\end{center}

\begin{center}
  \captionsetup{type=figure}
  \includegraphics[width=1.0\linewidth]{visualisatie ddt.png}
  \captionof{figure}[Visualisatie taak "contract maken" met toewijzing]{De taak wordt expliciet gemaakt voor deze medewerker en komt in zijn overzicht op hetzelfde moment.}
\end{center}

\begin{center}
  \captionsetup{type=figure}
  \includegraphics[width=0.75\linewidth]{log boma.png}
  \captionof{figure}[JSON-output slimme toewijzing taken]{De tweede taak uit vorige screenshot was eerst volgens de monitoring voor medewerker “Balthasar Boma”, maar de executie engine nam dit uit zijn handen omdat deze over zijn takenlimiet zou zitten.}
\end{center}

Indien het systeem bevestiging krijgt dat een taak voltooid of gefaald is, dan is deze uit het overzicht te filteren.
 
\begin{center}
  \captionsetup{type=figure}
  \includegraphics[width=1.0\linewidth]{takenoverzicht filtered.png}
  \captionof{figure}[Visualisatie takenoverzicht met filtering]{Het takenoverzicht met filtering op doorlopende taken.}
\end{center}

Het systeem registreert echter wel de status en moment van beëindiging. Gezien elke type taak via zijn taak referentie traceerbaar blijft, kan zo aan statistisch onderzoek en anomaliedetectie gedaan worden.
 
\begin{center}
  \captionsetup{type=figure}
  \includegraphics[width=1.0\linewidth]{takenoverzicht non-filtered.png}
  \captionof{figure}[Visualisatie takenoverzicht zonder filtering]{Het takenoverzicht met voltooide taken voor statistisch onderzoek en anomaliedetectie.}
\end{center}
%%=============================================================================
%% Conclusie
%%=============================================================================

\chapter{Conclusie}%
\label{ch:conclusie}

% TODO: Trek een duidelijke conclusie, in de vorm van een antwoord op de
% onderzoeksvra(a)g(en). Wat was jouw bijdrage aan het onderzoeksdomein en
% hoe biedt dit meerwaarde aan het vakgebied/doelgroep? 
% Reflecteer kritisch over het resultaat. In Engelse teksten wordt deze sectie
% ``Discussion'' genoemd. Had je deze uitkomst verwacht? Zijn er zaken die nog
% niet duidelijk zijn?
% Heeft het onderzoek geleid tot nieuwe vragen die uitnodigen tot verder 
%onderzoek?
\section{\IfLanguageName{dutch}{Resultaten}{Resultaten}}%
\label{sec:resultaten}
\subsection{\IfLanguageName{dutch}{Functionele vereisten}{Functionele vereisten}}%
\label{subsec:functionele vereisten}
\subsubsection{\IfLanguageName{dutch}{Monitoring}{Monitoring}}%
\label{subsubsec:monitoring}
De grootste use-cases van business zijn zeker behaald. Per lopende instantie van het simulatieproces is er een timeline met de huidige uitvoerder, de stappen en bereikte mijlpalen beschikbaar in sequentiële volgorde. Door deze informatie correct op te halen is het mogelijk om de klant in real-time te informeren van de status van zijn lopend proces in MyLiantis. Er zijn ook stappen gemaakt naar betere rapportage door de uitvoeringstijd per stap en eventuele afwijkingen in het normaal verloop zichtbaar te maken. Door dit af te nemen kan business intelligence opgezet worden om zo bottlenecks en anomalieën te detecteren op basis van servicelevel overeenkomsten. Het systeem weet ook hoeveel taken elke medewerker momenteel uitvoert en nog kan uitvoeren voor forecasting en balanceren van werklast. Dit is uiteraard relatief rudimentair, maar de technische haalbaarheid van het concept is zeker bewezen.
\subsubsection{\IfLanguageName{dutch}{Orkestratie}{Orkestratie}}%
\label{subsubsec:orkestratie}
In kader van orkestratie voldoet het systeem eveneens. Het genereert passende taken en zoekt de medewerker met het minst aantal actieve taken om deze uit te voeren tenzij expliciet aangewezen door de monitoring om dit niet te doen. In een latere iteratie kan dit verfijnd worden door bijvoorbeeld elke taak een granulair gewicht toe te kennen voor een betere balans. De API laat toe om alle huidige ingeplande taken voor een medewerker op te halen. Het is hierdoor mogelijk om een organische takenlijst voor te stellen in bijvoorbeeld de Digitale Cockpit. Verder kan het verloop van elke taak via de API doorgegeven worden opdat de data de werkelijkheid blijft volgen. Op de taak data kan eveneens rapportage gebouwd worden zodat bijvoorbeeld de doorlooptijd per type taak bepaald kan worden op data-gestuurde wijze.\newline
\subsection{\IfLanguageName{dutch}{Niet-functionele vereisten}{Niet-functionele vereisten}}%
\label{subsec:niet-functionele vereisten}
De proof-of-concept kan door zijn aard niet voldoen aan alle opgelegde functionele vereisten van het casusbedrijf gelet op het feit dat het geen toegang heeft tot de interne systemen. Het behalen van deze vereisten is een proof-of-concept dan ook geen must-have. De proof-of-concept voldoet echter wel aan volgende vereisten:
\begin{itemize}
  \item Het systeem is foutbestendig en gebouwd om zichzelf te remediëren na fouten.
  \item Het systeem voert alle synchrone bewerkingen uit aan 12 milliseconden of lager.
  \item Het kan een externe configuratie aanvaarden voor de mapping van de executie engine.
  \item De data is volledig generisch en bestaat louter uit referenties.
  \item Het systeem kan beveiligd worden via Spring Security.
  \item Het systeem genereert duidelijke logging bij elke handeling die aggregeerbaar is.
  \item Het systeem is volledig compliant met de interne technologie stack.
\end{itemize}

\subsection{\IfLanguageName{dutch}{Mogelijke verbeteringen}{Mogelijke verbeteringen}}%
\label{subsec:mogelijke verbeteringen}
Het project voldoet aan de vooropgestelde verwachtingen, maar elke systeem kan uiteraard iteratief verbeterd worden. Een eerste verbetering zou zijn om integraties bouwen met de masters rond processen en human resources. In de huidige opstelling is een singulier proces binnen scope, maar het zou elk proces moeten kunnen ondersteunen. Een integratie met een proces documentatietool zoals ARIS bij het casusbedrijf zou toelaten dat het systeem zijn executie engine dagelijks kan syncen.  Verder is de data van medewerkers en hun teams cruciaal om de juiste taken aan de juiste persoon te kunnen koppelen. \newline

Een tweede groot verbeterpunt is de inkomende monitoringdata uitbreiden en toegankelijker maken. Op basis van grondige analyse moet gekeken worden welke data het systeem zelf kan opvragen binnen de organisatie om de logs te verrijken en welke enkel het externe systeem kan leveren. Zodoende vergemakkelijkt de implementatie voor providers en verlaagt de load op hun systemen. Het beste scenario zou een plug-and-play library zijn die de providers implementeren en aanroepen om zo automatische de juiste data te formateren en door te sturen. \newline

Een derde groot verbeterpunt is het uitwerken van een extractie, transformatie en laadproces richting een datawarehouse. Het systeem zal naarmate meer processen toegevoegd worden aan de monitoring exponentieel meer data beginnen produceren die na ontsluiting van grote waarde is voor het bedrijf. Hoe sneller deze data kan landen bij de business intelligence experten, hoe hoger de business waarde van dit systeem zal zijn.

\section{\IfLanguageName{dutch}{Inhoudelijke Conclusie}{Inhoudelijke Conclusie}}%
\label{sec:inhoudelijke conclusie}
In dit werk werd onderzocht welk de vereisten zijn waaraan een systeem moet voldoen en welke software architectuur gepast is om custom proces monitoring en orkestratie te implementeren binnen een bedrijf met veel custom development applicaties. \newline

Allereerst werd een literatuurstudie gedaan om te begrijpen hoe hedendaagse systemen binnen het werkveld werken, wat hun mogelijkheden zijn en wat de best practices zijn bij het ontwerpen van dergelijk systeem. Op basis van een analyse binnen casusbedrijf Liantis die voldoet aan bovenstaande criteria werden hun vereisten omgezet in een lijst van functionele en niet-functionele vereisten die representatief zijn voor het bedrijf zelf als andere bedrijven van een gelijkaardig profiel.\newline

Verder werd een design gemaakt die architecturaal en functioneel voldoet aan de eisen van het casusbedrijf waarbij zowel proces- als systeem-agnostisch werd gewerkt om zo tegemoet te komen aan de technische en architecturale uitdagingen waar het bedrijf mee kampt door zijn veelvoud aan custom development.\newline

Dit design werd dan omgezet in een werkende proof-of-concept die gesimuleerde data van een proces kon verwerken in geconsolideerde executie logs die afnemers via een duidelijke API kunnen gebruiken om vereiste use-cases gevonden tijdens de analyse op te leveren.  Op basis van de data kon dit systeem ook taken aanmaken en toekennen aan medewerkers binnen het bedrijf. Op basis van een duidelijke API konden afnemers hiermee een taakoverzicht bouwen voor de betrokkenen medewerkers en die status van deze taken correct aanpassen op basis van hun verloop. Een rudimentaire versie van de realisaties die het casusbedrijf wil bouwen werd ook opgeleverd.\newline

Het resultaat toont daarmee hoe dergelijk systeem succesvol geïmplementeerd kan worden in een bedrijfsomgeving met veel custom development en vormt zo een blauwdruk waarmee een succesvolle proces monitoring en orkestratie bereikt kan worden dat reproduceerbaar is binnen soortgelijke bedrijven in het werkveld.\newline

Het doel van dit onderzoekswerk is daarmee bereikt waardoor de weg nu vrij is om dit systeem te implementeren bij het casusbedrijf. Hiervoor zijn er al reeds concrete stappen uitgevoerd dit kwartaal met uitzicht op verdere iteratieve uitbreiding in volgende kwartalen op basis van de reeks bereikte resultaten. Hiermee is het bijkomend doel van dit werk succesvol behaald en zal het werk uitgevoerd in kader van deze bachelor proef nog verder vervolgd worden in professionele setting.




%---------- Bijlagen -----------------------------------------------------------

\appendix

\chapter{Onderzoeksvoorstel}

Het onderwerp van deze bachelorproef is gebaseerd op een onderzoeksvoorstel dat vooraf werd beoordeeld door de promotor. Dat voorstel is opgenomen in deze bijlage.

%% TODO: 
\section{\IfLanguageName{dutch}{Abstract}{Abstract}}%
\label{sec:onderzoeksvoorstel abstract}
Dit werk onderzoekt hoe process monitoring en orchestration kan onderzocht en geïmplementeerd worden binnen een Belgisch bedrijf met een IT-context waarbij er veel custom development is. Het onderzoek kadert binnen de toenemende nood aan effectieve governance in complexe business-omgevingen, waar handmatige monitoring vaak ontoereikend is om problemen snel te detecteren en op te lossen. Verder kadert het zich in de groeiende vraag van bedrijven om hun medewerkers zo effectief mogelijk in te zetten via automatisatie. \newline

De centrale onderzoeksvraag is: \textbf{"Wat zijn de requirements waaraan een systeem moet voldoen om custom process monitoring en orchestration te implementeren binnen een bedrijf met veel custom development solutions?"}. Door het implementeren van geautomatiseerde monitoring- en orchestration-tools kunnen bedrijven processen real-time analyseren en bijsturen, wat resulteert in verhoogde betrouwbaarheid, flexibiliteit, transparantie richting de klant en naleving van compliance-vereisten. Dit onderzoek beoogt een praktisch raamwerk te ontwikkelen met proof-of-concept dat de toepassing van deze technologieën in een governance-structuur beschrijft en faciliteert. Verder zal het dit raamwerk praktisch toepassen op het bedrijf Liantis met een eerste proof-of-concept waarin een specifiek proces wordt ondersteund met monitoring en orchestration. Dit kan dan dienen als start voor uitbreiding naar verdere processen door een developmentteam. \newline

Methodologisch wordt een combinatie van literatuurstudie rond het domein, marktonderzoek en casestudie-analyse binnen het casusbedrijf Liantis gebruikt om het probleemdomein op te bouwen en een oplossing te generen. Er wordt gekeken naar de theorie en de markt om te onderzoeken wat er gangbaar is binnen het domein, waarnaar dit getoetst wordt aan de noden van Liantis voor een praktische functionele requirementsanalyse, technische analyse en niet-functionele analyse. \newline

Op basis van deze analyse wordt een theoretisch en proof-of-concept systeem ontworpen, waarna de proof-of-concept wordt gebouwd. Deze proof-of-concept zal dan toegepast worden een simulatie van een bestaand proces van het casusbedrijf om stakeholders te overtuigen van deze aanpak. Verwacht wordt dat de resultaten inzicht bieden in de impact van deze technologieën op procesbeheer, monitoring en orkestratie met een waardevolle proof-of-concept en blauwdruk voor business managers en stakeholders. De uitkomsten zal het bedrijf ondersteunen bij het optimaliseren van hun process governance, wat zal leiden tot efficiëntere processen en betere alignement met bedrijfsdoelstellingen.

% Verwijzing naar het bestand met de inhoud van het onderzoeksvoorstel
%---------- Inleiding ---------------------------------------------------------

\section{Inleiding}%
\label{sec:inleiding}

\subsection{Context}

Dit werk richt zich op het verbeteren van process governance door middel van process monitoring en orchestration, specifiek binnen het bedrijf Liantis, een grote Belgische onderneming actief in de HR-sector. Liantis biedt oplossingen in kader van payroll, preventie/welzijn op het werk en ondersteuning aan zelfstandigen. Omdat het bedrijf een wildgroei aan processen heeft die steeds complexer en dynamischer worden, heeft Liantis moeite om consistent en betrouwbaar beheer van zijn processen te waarborgen. Door zijn veelvoud aan producten is Liantis ook vaak aangewezen op custom development. Vooral door beperkte mogelijkheden om real-time inzicht te krijgen in de prestaties van zijn medewerkers binnen deze custom solutions, ontstaan er regelmatig inefficiënties en compliance-uitdagingen. Voor business-managers en analisten binnen het bedrijf vormt dit een knelpunt bij het waarborgen van de procesbetrouwbaarheid en het naleven van interne en externe regelgeving. 

\subsection{Onderzoeksvraag en Deelvragen}
De centrale onderzoeksvraag luidt dan ook \textbf{"Wat zijn de requirements waaraan een systeem moet voldoen om custom process monitoring en orchestration te implementeren binnen een bedrijf met veel custom development solutions?"} en dit praktisch toegepast binnen de specifieke case van Liantis. Door dit zodanig specifiek toe te passen kunnen we de deelvragen heel concreet maken voor zowel het probleemdomein als het oplossingsdomein.

\textbf{Deelvragen:}
\begin{itemize}
    \item \textbf{Probleemdomein:}
    \begin{enumerate}
        \item Wat zijn de theoretisch high-level requirements waaraan dit systeem moet voldoen?
        \item Wat zijn de mogelijkheden die bestaande process monitoring en orchestration software op de markt bieden?
        \item Wat zijn de specifieke noden van Liantis qua process monitoring en orchestration?
        \item Welke technische en architecturale uitdagingen zijn er bij de implementatie van dergelijke systemen in een bedrijfsomgeving met veel custom development?
    \end{enumerate}
    
    \item \textbf{Oplossingsdomein:}
    \begin{enumerate}
        \item Hoe implementeren we dergelijke systemen in een bedrijfsomgeving met veel custom development zoals Liantis?
        \item Aan welke criteria moet dergelijk systeem voldoen om succesvolle process monitoring en orchestration te bereiken binnen het bedrijf?
    \end{enumerate}
\end{itemize}

\subsection{Doel en Resultaat}

Het doel van deze bachelorproef is om een analyse te maken van process monitoring en orchestration binnen de huidige IT-markt en dit naast de specifieke noden van Liantis te zetten om zo een solution te maken voldoet voor hun doeleinden. Verder zal het een proof-of-concept realiseren in de vorm van een integreerbare library die in zowel legacy als nieuwe solutions kan werken en die procesmonitoring logs stuurt naar een tool die dit allemaal bijhoudt in leesbare vorm waarop rapportage kan gedaan worden en die in kader van orchestration taken kan genereren voor medewerkers. We gaan deze proof-of-concept dan toepassen op een specifiek gekozen proces om data te genereren. Dit zal dan gebruikt worden om de relevante stakeholders te overtuigen om dit verder uit te bouwen en uit te rollen naar andere processen.  Hiervoor zal eerst een uitgebreide literatuurstudie worden uitgevoerd om inzicht te krijgen in bestaande methodieken en theoretische achterslag op het gebied van process monitoring en orchestration. Dit wordt dan gevolgd door een marktonderzoek waarbij we de mogelijkheden van bestaande tools op de markt bekijken. Vervolgens wordt er een casestudie uitgevoerd binnen het bedrijf om de huidige knelpunten en noden in kaart te brengen. Op basis hiervan zal een proof-of-concept ontworpen en ontwikkeld worden waarbij we aantonen hoe dit een praktisch en specifiek gekozen bedrijfsproces zal ondersteunen. Het beoogde eindresultaat voor een succesvolle bachelorproef bestaat uit een gedetailleerd implementatieplan met proof-of-concept dat de IT-afdeling van Liantis helpt om monitoring en orchestration op een effectieve manier in de governance-structuur te integreren.

%---------- Stand van zaken ---------------------------------------------------

\section{Stand Van Zaken}%
\label{sec:stand_van_zaken}
\subsection{Business Process Management en Governance}

Het domein van business process management en governance richt zich op het verbeteren van bedrijfsprocessen door middel van monitoring, analyse, en optimalisatie. BPM biedt organisaties de mogelijkheid om processen te stroomlijnen en deze af te stemmen op de strategische doelen van de organisatie \autocite{Dumas2018}. Binnen deze context spelen process monitoring en orchestration een cruciale rol. Deze technologieën maken het mogelijk om bedrijfsprocessen op een efficiënte en geautomatiseerde manier te beheren en aan te passen \autocite{Weske2019}.

\subsection{Monitoring en Orchestration}

Process monitoring is essentieel binnen BPM omdat het organisaties near-realtime inzicht geeft in de prestaties van hun bedrijfsprocessen, waardoor afwijkingen of inefficiënties snel kunnen worden geïdentificeerd \autocite{Janiesch2012}. Dit is dan voedingsbodem voor process mining, waarbij we de logs uit de monitoring gebruiken om tot analytische inzichten te komen. Process mining biedt niet alleen inzicht in het huidige verloop van processen, maar maakt ook een data-gedreven optimalisatie mogelijk door onregelmatigheden bloot te leggen die vaak onzichtbaar blijven bij traditionele monitoring \autocite{Aalst2016}. Daarnaast kan Business Activity Monitoring organisaties helpen om niet alleen retrospectief, maar ook proactief op afwijkingen te reageren, wat waardevol is voor processen met hoge compliance-eisen \autocite{Janiesch2012}. Process orchestration gaat een stap verder dan monitoring door bedrijfsprocessen te automatiseren en te coördineren op basis van vastgelegde beleidslijnen en workflows. Dit is vooral nuttig voor organisaties die streven naar consistentie en compliance in hun bedrijfsvoering \autocite{Weske2019}. 

\subsection{Open Vragen}

De open vraag die heeft geleid tot dit onderzoek is de technische uitdaging van het implementeren van deze systemen binnen een bedrijf zoals Liantis met veel custom development. Net zoals veel bedrijven is Liantis immers organisch gegroeid en bevat het veel legacy inhouse solutions die op elkaar inspelen. Dergelijke monitoring en orkestratie systemen moeten op een agnostische manier kunnen omgaan met de output van allerhande processen en systemen. De implementatie van deze systemen in een bedrijfscontext brengt dus verschillende technische uitdagingen met zich mee die vaak opnieuw custom oplossingen vereisen. Vakliteratuur geeft een aantal interessante opties zoals het gebruik van middleware \autocite{Weber2018}, maar er is een zeer duidelijke lacune binnen het domein dat interessant zou zijn om te onderzoeken. 

\subsection{Vergelijkbare Studies en Unieke Waarde}

De waarde van dit domein is al vast en zeker bewezen door vergelijkbare studies binnen grote organisaties \autocite{Harmon2014} die benadrukken dat de implementatie van geavanceerde monitoring- en orchestration technieken helpen om consistent aan governance- en compliance-vereisten te voldoen. Dit onderzoek biedt echter een unieke bijdrage door zich te richten op de praktische toepasbaarheid van deze technieken bij een groot Belgische bedrijf wiens IT-context breed toepasbaar is. Door dit uit te werken zal er een praktisch raamwerk bestaan voor de implementatie van dergelijke systemen binnen een bedrijf met dergelijk profiel met proof-of-concept dat kan dienen als blauwdruk voor veel Belgische bedrijven. 

%---------- Methodologie ------------------------------------------------------
\section{Methodologie}%
\label{sec:methodologie}

Om de onderzoeksvraag te beantwoorden en een effectieve aanpak te ontwikkelen voor de implementatie van process monitoring en orchestration binnen Liantis, wordt gebruik gemaakt van een gefaseerde onderzoeksmethodologie die literatuurstudie, marktonderzoek, case studieanalyse, requirements-analyse en een proof-of-concept ontwikkeling omvat. Deze combinatie van technieken waarborgt zowel theoretische onderbouwing als technische diepgang, wat essentieel is voor het creëren van een valide en werkbaar implementatiemodel.

\subsection{Fase 1: Literatuurstudie (4 weken) (Probleemdomein)}

De eerste fase bestaat uit een gedetailleerde literatuurstudie en marktonderzoek. Hierbij worden academische artikelen en technische bronnen over business process management, monitoring, orchestration, en implementatie-uitdagingen grondig geanalyseerd. Deze studie heeft als doel om een diepgaand inzicht te verkrijgen in de huidige stand van zaken van het domein, de mogelijkheden binnen de huidige markt, de typische uitdagingen en succesfactoren binnen BPM-implementaties voor process governance. De bevindingen uit de literatuurstudie vormen de theoretische basis van het onderzoek en dienen als referentiekader voor de verdere fasen. \\

\textbf{Deliverable:} een theoretisch component met gedocumenteerde inzichten over BPM-standaarden, technische obstakels en oplossingen op de markt voor de organisatie.

\subsection{Fase 2: Requirements-analyse binnen Liantis (2 weken) (Probleemdomein)}

Om de specifieke behoeften en uitdagingen van de casusorganisatie vast te stellen, wordt een requirements-analyse uitgevoerd door middel van gestructureerde workshops met stakeholders binnen de organisatie, zoals IT-managers, procesanalisten en compliance-officers. Verder wordt de bestaande IT-infrastructuur bekeken om te kijken hoe we dit het beste inpassen om de impact minimaal te houden. We bepalen hier ook het complexe proces dat in scope zal zijn voor onze proof-of-concept. \\

\textbf{Deliverable:} een requirementsrapport met gedetailleerde, gevalideerde vereisten voor de implementatie.

\subsection{Fase 3: Technische en Functionele Analyse (2 weken) (Oplossingsdomein)}

Na de requirements-analyse wordt een technische analyse uitgevoerd om geschikte tools en technologieën te selecteren. Hierbij zoeken we naar een proces documentatie tool waarmee we kunnen integreren als bron van waarheid, modelleren we een architectuur voor deze systemen, schrijven we een contract uit dat alle systemen gaan gebruiken voor de monitoring en orkestratie en voeren we de analyse uit voor de proof-of-concept.  \\

\textbf{Deliverable:} design documenten voor het process monitoring en orchestration systeem met implementatieplan.

\subsection{Fase 4: Proof-of-Concept Ontwikkeling en Validatie (4 weken) (Oplossingsdomein)}

In deze fase wordt een proof-of-concept ontwikkeld om de haalbaarheid van het voorgestelde implementatiemodel te testen. De proof-of-concept omvat de integratie van het monitoring- en orchestration systeem met een voorbeeldproces van de Liantis. Dit proces wordt gesimuleerd met representatieve data om de configuratie van monitoring en event-triggered orchestration te testen. Uit de data zullen we dan eerste conclusies en rapportage kunnen opbouwen voor verdere iteratieve verbeteringen en aanpassingen. Deze proof-of-concept zal dan gebruikt worden om stakeholders te overtuigen en als blauwdruk dienen voor de implementatie van dit systeem bij verdere processen. \\

\textbf{Deliverable:} een gedetailleerd proof-of-concept rapport en werkend prototype dat het systeem demonstreert.

%---------- Verwachte resultaten ----------------------------------------------
\section{Verwachtingen}%
\label{sec:verwachtingen}
De verwachting binnen dit onderzoek is dat er een implementatieplan zal ontstaan die toelaat om process governance te waarborgen binnen Liantis door te steunen op een sterk framework van monitoring en orchestration met een proof-of-concept om stakeholders te overtuigen. Dit zal dan fungeren als verdere voedingsbodem voor het uitwerken van dit systeem binnen de organisatie met als gevolg consistentere processen die makkelijker geoptimaliseerd kunnen worden.

Verder is het mijn verwachting dat de resultaten uit dit onderzoek toepasbaar zullen zijn op bedrijven die een gelijkaardig profiel hebben als Liantis. Liantis is binnen zijn IT-context zeker geen uitzondering en het is mijn doel om binnen dit onderzoek Liantis als casestudy te gebruiken om zo een framework te ontwikkelen waarmee veel meer bedrijven gediend zijn.



%%---------- Andere bijlagen --------------------------------------------------
% TODO: Voeg hier eventuele andere bijlagen toe. Bv. als je deze BP voor de
% tweede keer indient, een overzicht van de verbeteringen t.o.v. het origineel.
%\input{...}

De code van de proof-of-concept monitoring en orkestratie software is hier ingediend als bijlage.
%%=============================================================================
%% Cpde
%%=============================================================================

\chapter{\IfLanguageName{dutch}{Code}{Code}}%
\label{ch:code}

%%---------- Backmatter, referentielijst ---------------------------------------

\backmatter{}

\setlength\bibitemsep{2pt} %% Add Some space between the bibliograpy entries
\printbibliography[heading=bibintoc]

\end{document}
